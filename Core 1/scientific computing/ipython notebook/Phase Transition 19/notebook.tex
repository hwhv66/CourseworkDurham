
% Default to the notebook output style

    


% Inherit from the specified cell style.




    
\documentclass[11pt]{article}

    
    
    \usepackage[T1]{fontenc}
    % Nicer default font (+ math font) than Computer Modern for most use cases
    \usepackage{mathpazo}

    % Basic figure setup, for now with no caption control since it's done
    % automatically by Pandoc (which extracts ![](path) syntax from Markdown).
    \usepackage{graphicx}
    % We will generate all images so they have a width \maxwidth. This means
    % that they will get their normal width if they fit onto the page, but
    % are scaled down if they would overflow the margins.
    \makeatletter
    \def\maxwidth{\ifdim\Gin@nat@width>\linewidth\linewidth
    \else\Gin@nat@width\fi}
    \makeatother
    \let\Oldincludegraphics\includegraphics
    % Set max figure width to be 80% of text width, for now hardcoded.
    \renewcommand{\includegraphics}[1]{\Oldincludegraphics[width=.8\maxwidth]{#1}}
    % Ensure that by default, figures have no caption (until we provide a
    % proper Figure object with a Caption API and a way to capture that
    % in the conversion process - todo).
    \usepackage{caption}
    \DeclareCaptionLabelFormat{nolabel}{}
    \captionsetup{labelformat=nolabel}

    \usepackage{adjustbox} % Used to constrain images to a maximum size 
    \usepackage{xcolor} % Allow colors to be defined
    \usepackage{enumerate} % Needed for markdown enumerations to work
    \usepackage{geometry} % Used to adjust the document margins
    \usepackage{amsmath} % Equations
    \usepackage{amssymb} % Equations
    \usepackage{textcomp} % defines textquotesingle
    % Hack from http://tex.stackexchange.com/a/47451/13684:
    \AtBeginDocument{%
        \def\PYZsq{\textquotesingle}% Upright quotes in Pygmentized code
    }
    \usepackage{upquote} % Upright quotes for verbatim code
    \usepackage{eurosym} % defines \euro
    \usepackage[mathletters]{ucs} % Extended unicode (utf-8) support
    \usepackage[utf8x]{inputenc} % Allow utf-8 characters in the tex document
    \usepackage{fancyvrb} % verbatim replacement that allows latex
    \usepackage{grffile} % extends the file name processing of package graphics 
                         % to support a larger range 
    % The hyperref package gives us a pdf with properly built
    % internal navigation ('pdf bookmarks' for the table of contents,
    % internal cross-reference links, web links for URLs, etc.)
    \usepackage{hyperref}
    \usepackage{longtable} % longtable support required by pandoc >1.10
    \usepackage{booktabs}  % table support for pandoc > 1.12.2
    \usepackage[inline]{enumitem} % IRkernel/repr support (it uses the enumerate* environment)
    \usepackage[normalem]{ulem} % ulem is needed to support strikethroughs (\sout)
                                % normalem makes italics be italics, not underlines
    

    
    
    % Colors for the hyperref package
    \definecolor{urlcolor}{rgb}{0,.145,.698}
    \definecolor{linkcolor}{rgb}{.71,0.21,0.01}
    \definecolor{citecolor}{rgb}{.12,.54,.11}

    % ANSI colors
    \definecolor{ansi-black}{HTML}{3E424D}
    \definecolor{ansi-black-intense}{HTML}{282C36}
    \definecolor{ansi-red}{HTML}{E75C58}
    \definecolor{ansi-red-intense}{HTML}{B22B31}
    \definecolor{ansi-green}{HTML}{00A250}
    \definecolor{ansi-green-intense}{HTML}{007427}
    \definecolor{ansi-yellow}{HTML}{DDB62B}
    \definecolor{ansi-yellow-intense}{HTML}{B27D12}
    \definecolor{ansi-blue}{HTML}{208FFB}
    \definecolor{ansi-blue-intense}{HTML}{0065CA}
    \definecolor{ansi-magenta}{HTML}{D160C4}
    \definecolor{ansi-magenta-intense}{HTML}{A03196}
    \definecolor{ansi-cyan}{HTML}{60C6C8}
    \definecolor{ansi-cyan-intense}{HTML}{258F8F}
    \definecolor{ansi-white}{HTML}{C5C1B4}
    \definecolor{ansi-white-intense}{HTML}{A1A6B2}

    % commands and environments needed by pandoc snippets
    % extracted from the output of `pandoc -s`
    \providecommand{\tightlist}{%
      \setlength{\itemsep}{0pt}\setlength{\parskip}{0pt}}
    \DefineVerbatimEnvironment{Highlighting}{Verbatim}{commandchars=\\\{\}}
    % Add ',fontsize=\small' for more characters per line
    \newenvironment{Shaded}{}{}
    \newcommand{\KeywordTok}[1]{\textcolor[rgb]{0.00,0.44,0.13}{\textbf{{#1}}}}
    \newcommand{\DataTypeTok}[1]{\textcolor[rgb]{0.56,0.13,0.00}{{#1}}}
    \newcommand{\DecValTok}[1]{\textcolor[rgb]{0.25,0.63,0.44}{{#1}}}
    \newcommand{\BaseNTok}[1]{\textcolor[rgb]{0.25,0.63,0.44}{{#1}}}
    \newcommand{\FloatTok}[1]{\textcolor[rgb]{0.25,0.63,0.44}{{#1}}}
    \newcommand{\CharTok}[1]{\textcolor[rgb]{0.25,0.44,0.63}{{#1}}}
    \newcommand{\StringTok}[1]{\textcolor[rgb]{0.25,0.44,0.63}{{#1}}}
    \newcommand{\CommentTok}[1]{\textcolor[rgb]{0.38,0.63,0.69}{\textit{{#1}}}}
    \newcommand{\OtherTok}[1]{\textcolor[rgb]{0.00,0.44,0.13}{{#1}}}
    \newcommand{\AlertTok}[1]{\textcolor[rgb]{1.00,0.00,0.00}{\textbf{{#1}}}}
    \newcommand{\FunctionTok}[1]{\textcolor[rgb]{0.02,0.16,0.49}{{#1}}}
    \newcommand{\RegionMarkerTok}[1]{{#1}}
    \newcommand{\ErrorTok}[1]{\textcolor[rgb]{1.00,0.00,0.00}{\textbf{{#1}}}}
    \newcommand{\NormalTok}[1]{{#1}}
    
    % Additional commands for more recent versions of Pandoc
    \newcommand{\ConstantTok}[1]{\textcolor[rgb]{0.53,0.00,0.00}{{#1}}}
    \newcommand{\SpecialCharTok}[1]{\textcolor[rgb]{0.25,0.44,0.63}{{#1}}}
    \newcommand{\VerbatimStringTok}[1]{\textcolor[rgb]{0.25,0.44,0.63}{{#1}}}
    \newcommand{\SpecialStringTok}[1]{\textcolor[rgb]{0.73,0.40,0.53}{{#1}}}
    \newcommand{\ImportTok}[1]{{#1}}
    \newcommand{\DocumentationTok}[1]{\textcolor[rgb]{0.73,0.13,0.13}{\textit{{#1}}}}
    \newcommand{\AnnotationTok}[1]{\textcolor[rgb]{0.38,0.63,0.69}{\textbf{\textit{{#1}}}}}
    \newcommand{\CommentVarTok}[1]{\textcolor[rgb]{0.38,0.63,0.69}{\textbf{\textit{{#1}}}}}
    \newcommand{\VariableTok}[1]{\textcolor[rgb]{0.10,0.09,0.49}{{#1}}}
    \newcommand{\ControlFlowTok}[1]{\textcolor[rgb]{0.00,0.44,0.13}{\textbf{{#1}}}}
    \newcommand{\OperatorTok}[1]{\textcolor[rgb]{0.40,0.40,0.40}{{#1}}}
    \newcommand{\BuiltInTok}[1]{{#1}}
    \newcommand{\ExtensionTok}[1]{{#1}}
    \newcommand{\PreprocessorTok}[1]{\textcolor[rgb]{0.74,0.48,0.00}{{#1}}}
    \newcommand{\AttributeTok}[1]{\textcolor[rgb]{0.49,0.56,0.16}{{#1}}}
    \newcommand{\InformationTok}[1]{\textcolor[rgb]{0.38,0.63,0.69}{\textbf{\textit{{#1}}}}}
    \newcommand{\WarningTok}[1]{\textcolor[rgb]{0.38,0.63,0.69}{\textbf{\textit{{#1}}}}}
    
    
    % Define a nice break command that doesn't care if a line doesn't already
    % exist.
    \def\br{\hspace*{\fill} \\* }
    % Math Jax compatability definitions
    \def\gt{>}
    \def\lt{<}
    % Document parameters
    \title{phase\_transition}
    
    
    

    % Pygments definitions
    
\makeatletter
\def\PY@reset{\let\PY@it=\relax \let\PY@bf=\relax%
    \let\PY@ul=\relax \let\PY@tc=\relax%
    \let\PY@bc=\relax \let\PY@ff=\relax}
\def\PY@tok#1{\csname PY@tok@#1\endcsname}
\def\PY@toks#1+{\ifx\relax#1\empty\else%
    \PY@tok{#1}\expandafter\PY@toks\fi}
\def\PY@do#1{\PY@bc{\PY@tc{\PY@ul{%
    \PY@it{\PY@bf{\PY@ff{#1}}}}}}}
\def\PY#1#2{\PY@reset\PY@toks#1+\relax+\PY@do{#2}}

\expandafter\def\csname PY@tok@w\endcsname{\def\PY@tc##1{\textcolor[rgb]{0.73,0.73,0.73}{##1}}}
\expandafter\def\csname PY@tok@c\endcsname{\let\PY@it=\textit\def\PY@tc##1{\textcolor[rgb]{0.25,0.50,0.50}{##1}}}
\expandafter\def\csname PY@tok@cp\endcsname{\def\PY@tc##1{\textcolor[rgb]{0.74,0.48,0.00}{##1}}}
\expandafter\def\csname PY@tok@k\endcsname{\let\PY@bf=\textbf\def\PY@tc##1{\textcolor[rgb]{0.00,0.50,0.00}{##1}}}
\expandafter\def\csname PY@tok@kp\endcsname{\def\PY@tc##1{\textcolor[rgb]{0.00,0.50,0.00}{##1}}}
\expandafter\def\csname PY@tok@kt\endcsname{\def\PY@tc##1{\textcolor[rgb]{0.69,0.00,0.25}{##1}}}
\expandafter\def\csname PY@tok@o\endcsname{\def\PY@tc##1{\textcolor[rgb]{0.40,0.40,0.40}{##1}}}
\expandafter\def\csname PY@tok@ow\endcsname{\let\PY@bf=\textbf\def\PY@tc##1{\textcolor[rgb]{0.67,0.13,1.00}{##1}}}
\expandafter\def\csname PY@tok@nb\endcsname{\def\PY@tc##1{\textcolor[rgb]{0.00,0.50,0.00}{##1}}}
\expandafter\def\csname PY@tok@nf\endcsname{\def\PY@tc##1{\textcolor[rgb]{0.00,0.00,1.00}{##1}}}
\expandafter\def\csname PY@tok@nc\endcsname{\let\PY@bf=\textbf\def\PY@tc##1{\textcolor[rgb]{0.00,0.00,1.00}{##1}}}
\expandafter\def\csname PY@tok@nn\endcsname{\let\PY@bf=\textbf\def\PY@tc##1{\textcolor[rgb]{0.00,0.00,1.00}{##1}}}
\expandafter\def\csname PY@tok@ne\endcsname{\let\PY@bf=\textbf\def\PY@tc##1{\textcolor[rgb]{0.82,0.25,0.23}{##1}}}
\expandafter\def\csname PY@tok@nv\endcsname{\def\PY@tc##1{\textcolor[rgb]{0.10,0.09,0.49}{##1}}}
\expandafter\def\csname PY@tok@no\endcsname{\def\PY@tc##1{\textcolor[rgb]{0.53,0.00,0.00}{##1}}}
\expandafter\def\csname PY@tok@nl\endcsname{\def\PY@tc##1{\textcolor[rgb]{0.63,0.63,0.00}{##1}}}
\expandafter\def\csname PY@tok@ni\endcsname{\let\PY@bf=\textbf\def\PY@tc##1{\textcolor[rgb]{0.60,0.60,0.60}{##1}}}
\expandafter\def\csname PY@tok@na\endcsname{\def\PY@tc##1{\textcolor[rgb]{0.49,0.56,0.16}{##1}}}
\expandafter\def\csname PY@tok@nt\endcsname{\let\PY@bf=\textbf\def\PY@tc##1{\textcolor[rgb]{0.00,0.50,0.00}{##1}}}
\expandafter\def\csname PY@tok@nd\endcsname{\def\PY@tc##1{\textcolor[rgb]{0.67,0.13,1.00}{##1}}}
\expandafter\def\csname PY@tok@s\endcsname{\def\PY@tc##1{\textcolor[rgb]{0.73,0.13,0.13}{##1}}}
\expandafter\def\csname PY@tok@sd\endcsname{\let\PY@it=\textit\def\PY@tc##1{\textcolor[rgb]{0.73,0.13,0.13}{##1}}}
\expandafter\def\csname PY@tok@si\endcsname{\let\PY@bf=\textbf\def\PY@tc##1{\textcolor[rgb]{0.73,0.40,0.53}{##1}}}
\expandafter\def\csname PY@tok@se\endcsname{\let\PY@bf=\textbf\def\PY@tc##1{\textcolor[rgb]{0.73,0.40,0.13}{##1}}}
\expandafter\def\csname PY@tok@sr\endcsname{\def\PY@tc##1{\textcolor[rgb]{0.73,0.40,0.53}{##1}}}
\expandafter\def\csname PY@tok@ss\endcsname{\def\PY@tc##1{\textcolor[rgb]{0.10,0.09,0.49}{##1}}}
\expandafter\def\csname PY@tok@sx\endcsname{\def\PY@tc##1{\textcolor[rgb]{0.00,0.50,0.00}{##1}}}
\expandafter\def\csname PY@tok@m\endcsname{\def\PY@tc##1{\textcolor[rgb]{0.40,0.40,0.40}{##1}}}
\expandafter\def\csname PY@tok@gh\endcsname{\let\PY@bf=\textbf\def\PY@tc##1{\textcolor[rgb]{0.00,0.00,0.50}{##1}}}
\expandafter\def\csname PY@tok@gu\endcsname{\let\PY@bf=\textbf\def\PY@tc##1{\textcolor[rgb]{0.50,0.00,0.50}{##1}}}
\expandafter\def\csname PY@tok@gd\endcsname{\def\PY@tc##1{\textcolor[rgb]{0.63,0.00,0.00}{##1}}}
\expandafter\def\csname PY@tok@gi\endcsname{\def\PY@tc##1{\textcolor[rgb]{0.00,0.63,0.00}{##1}}}
\expandafter\def\csname PY@tok@gr\endcsname{\def\PY@tc##1{\textcolor[rgb]{1.00,0.00,0.00}{##1}}}
\expandafter\def\csname PY@tok@ge\endcsname{\let\PY@it=\textit}
\expandafter\def\csname PY@tok@gs\endcsname{\let\PY@bf=\textbf}
\expandafter\def\csname PY@tok@gp\endcsname{\let\PY@bf=\textbf\def\PY@tc##1{\textcolor[rgb]{0.00,0.00,0.50}{##1}}}
\expandafter\def\csname PY@tok@go\endcsname{\def\PY@tc##1{\textcolor[rgb]{0.53,0.53,0.53}{##1}}}
\expandafter\def\csname PY@tok@gt\endcsname{\def\PY@tc##1{\textcolor[rgb]{0.00,0.27,0.87}{##1}}}
\expandafter\def\csname PY@tok@err\endcsname{\def\PY@bc##1{\setlength{\fboxsep}{0pt}\fcolorbox[rgb]{1.00,0.00,0.00}{1,1,1}{\strut ##1}}}
\expandafter\def\csname PY@tok@kc\endcsname{\let\PY@bf=\textbf\def\PY@tc##1{\textcolor[rgb]{0.00,0.50,0.00}{##1}}}
\expandafter\def\csname PY@tok@kd\endcsname{\let\PY@bf=\textbf\def\PY@tc##1{\textcolor[rgb]{0.00,0.50,0.00}{##1}}}
\expandafter\def\csname PY@tok@kn\endcsname{\let\PY@bf=\textbf\def\PY@tc##1{\textcolor[rgb]{0.00,0.50,0.00}{##1}}}
\expandafter\def\csname PY@tok@kr\endcsname{\let\PY@bf=\textbf\def\PY@tc##1{\textcolor[rgb]{0.00,0.50,0.00}{##1}}}
\expandafter\def\csname PY@tok@bp\endcsname{\def\PY@tc##1{\textcolor[rgb]{0.00,0.50,0.00}{##1}}}
\expandafter\def\csname PY@tok@fm\endcsname{\def\PY@tc##1{\textcolor[rgb]{0.00,0.00,1.00}{##1}}}
\expandafter\def\csname PY@tok@vc\endcsname{\def\PY@tc##1{\textcolor[rgb]{0.10,0.09,0.49}{##1}}}
\expandafter\def\csname PY@tok@vg\endcsname{\def\PY@tc##1{\textcolor[rgb]{0.10,0.09,0.49}{##1}}}
\expandafter\def\csname PY@tok@vi\endcsname{\def\PY@tc##1{\textcolor[rgb]{0.10,0.09,0.49}{##1}}}
\expandafter\def\csname PY@tok@vm\endcsname{\def\PY@tc##1{\textcolor[rgb]{0.10,0.09,0.49}{##1}}}
\expandafter\def\csname PY@tok@sa\endcsname{\def\PY@tc##1{\textcolor[rgb]{0.73,0.13,0.13}{##1}}}
\expandafter\def\csname PY@tok@sb\endcsname{\def\PY@tc##1{\textcolor[rgb]{0.73,0.13,0.13}{##1}}}
\expandafter\def\csname PY@tok@sc\endcsname{\def\PY@tc##1{\textcolor[rgb]{0.73,0.13,0.13}{##1}}}
\expandafter\def\csname PY@tok@dl\endcsname{\def\PY@tc##1{\textcolor[rgb]{0.73,0.13,0.13}{##1}}}
\expandafter\def\csname PY@tok@s2\endcsname{\def\PY@tc##1{\textcolor[rgb]{0.73,0.13,0.13}{##1}}}
\expandafter\def\csname PY@tok@sh\endcsname{\def\PY@tc##1{\textcolor[rgb]{0.73,0.13,0.13}{##1}}}
\expandafter\def\csname PY@tok@s1\endcsname{\def\PY@tc##1{\textcolor[rgb]{0.73,0.13,0.13}{##1}}}
\expandafter\def\csname PY@tok@mb\endcsname{\def\PY@tc##1{\textcolor[rgb]{0.40,0.40,0.40}{##1}}}
\expandafter\def\csname PY@tok@mf\endcsname{\def\PY@tc##1{\textcolor[rgb]{0.40,0.40,0.40}{##1}}}
\expandafter\def\csname PY@tok@mh\endcsname{\def\PY@tc##1{\textcolor[rgb]{0.40,0.40,0.40}{##1}}}
\expandafter\def\csname PY@tok@mi\endcsname{\def\PY@tc##1{\textcolor[rgb]{0.40,0.40,0.40}{##1}}}
\expandafter\def\csname PY@tok@il\endcsname{\def\PY@tc##1{\textcolor[rgb]{0.40,0.40,0.40}{##1}}}
\expandafter\def\csname PY@tok@mo\endcsname{\def\PY@tc##1{\textcolor[rgb]{0.40,0.40,0.40}{##1}}}
\expandafter\def\csname PY@tok@ch\endcsname{\let\PY@it=\textit\def\PY@tc##1{\textcolor[rgb]{0.25,0.50,0.50}{##1}}}
\expandafter\def\csname PY@tok@cm\endcsname{\let\PY@it=\textit\def\PY@tc##1{\textcolor[rgb]{0.25,0.50,0.50}{##1}}}
\expandafter\def\csname PY@tok@cpf\endcsname{\let\PY@it=\textit\def\PY@tc##1{\textcolor[rgb]{0.25,0.50,0.50}{##1}}}
\expandafter\def\csname PY@tok@c1\endcsname{\let\PY@it=\textit\def\PY@tc##1{\textcolor[rgb]{0.25,0.50,0.50}{##1}}}
\expandafter\def\csname PY@tok@cs\endcsname{\let\PY@it=\textit\def\PY@tc##1{\textcolor[rgb]{0.25,0.50,0.50}{##1}}}

\def\PYZbs{\char`\\}
\def\PYZus{\char`\_}
\def\PYZob{\char`\{}
\def\PYZcb{\char`\}}
\def\PYZca{\char`\^}
\def\PYZam{\char`\&}
\def\PYZlt{\char`\<}
\def\PYZgt{\char`\>}
\def\PYZsh{\char`\#}
\def\PYZpc{\char`\%}
\def\PYZdl{\char`\$}
\def\PYZhy{\char`\-}
\def\PYZsq{\char`\'}
\def\PYZdq{\char`\"}
\def\PYZti{\char`\~}
% for compatibility with earlier versions
\def\PYZat{@}
\def\PYZlb{[}
\def\PYZrb{]}
\makeatother


    % Exact colors from NB
    \definecolor{incolor}{rgb}{0.0, 0.0, 0.5}
    \definecolor{outcolor}{rgb}{0.545, 0.0, 0.0}



    
    % Prevent overflowing lines due to hard-to-break entities
    \sloppy 
    % Setup hyperref package
    \hypersetup{
      breaklinks=true,  % so long urls are correctly broken across lines
      colorlinks=true,
      urlcolor=urlcolor,
      linkcolor=linkcolor,
      citecolor=citecolor,
      }
    % Slightly bigger margins than the latex defaults
    
    \geometry{verbose,tmargin=1in,bmargin=1in,lmargin=1in,rmargin=1in}
    
    

    \begin{document}
    
    
    \maketitle
    
    

    
    \hypertarget{the-ising-model-and-phase-transitions}{%
\section{The Ising model and phase
transitions}\label{the-ising-model-and-phase-transitions}}

    \hypertarget{remarks-on-completing-the-module}{%
\subsubsection{Remarks on completing the
module}\label{remarks-on-completing-the-module}}

This assignment is summatively assessed. It is imperative that you
submit the notebook on time.

    \begin{Verbatim}[commandchars=\\\{\}]
{\color{incolor}In [{\color{incolor} }]:} \PY{c+c1}{\PYZsh{}\PYZsh{}\PYZsh{}\PYZsh{}\PYZsh{} from ipywidgets import widgets, interact, interactive, fixed}
        \PY{k+kn}{from} \PY{n+nn}{ipywidgets} \PY{k}{import} \PY{n}{widgets}\PY{p}{,} \PY{n}{interact}\PY{p}{,} \PY{n}{interactive}\PY{p}{,} \PY{n}{fixed}
        \PY{k+kn}{from} \PY{n+nn}{ipywidgets} \PY{k}{import} \PY{n}{Button}\PY{p}{,} \PY{n}{HBox}\PY{p}{,} \PY{n}{VBox}
        \PY{k+kn}{import} \PY{n+nn}{shelve}
        \PY{n}{assessmentName}\PY{o}{=}\PY{l+s+s2}{\PYZdq{}}\PY{l+s+s2}{ID}\PY{l+s+s2}{\PYZdq{}}\PY{p}{;}
        \PY{k+kn}{import} \PY{n+nn}{os}
        
        \PY{k}{def} \PY{n+nf}{get\PYZus{}last\PYZus{}value}\PY{p}{(}\PY{n}{key}\PY{p}{)}\PY{p}{:}
            \PY{k}{if} \PY{n}{os}\PY{o}{.}\PY{n}{path}\PY{o}{.}\PY{n}{isfile}\PY{p}{(}\PY{l+s+s1}{\PYZsq{}}\PY{l+s+s1}{.choices.shelve}\PY{l+s+s1}{\PYZsq{}}\PY{p}{)} \PY{o+ow}{or} \PY{n}{os}\PY{o}{.}\PY{n}{path}\PY{o}{.}\PY{n}{isfile}\PY{p}{(}\PY{l+s+s1}{\PYZsq{}}\PY{l+s+s1}{.choices.shelve.dir}\PY{l+s+s1}{\PYZsq{}}\PY{p}{)}\PY{p}{:}
                \PY{n}{s}\PY{o}{=}\PY{n}{shelve}\PY{o}{.}\PY{n}{open}\PY{p}{(}\PY{l+s+s1}{\PYZsq{}}\PY{l+s+s1}{.choices.shelve}\PY{l+s+s1}{\PYZsq{}}\PY{p}{)}
                \PY{k}{return} \PY{n}{s}\PY{o}{.}\PY{n}{get}\PY{p}{(}\PY{n}{key}\PY{p}{,}\PY{k+kc}{None}\PY{p}{)}
            \PY{k}{return} \PY{k+kc}{None}
        
        \PY{k}{def} \PY{n+nf}{make\PYZus{}value\PYZus{}change\PYZus{}fn}\PY{p}{(}\PY{n}{assessmentName}\PY{p}{,}\PY{n}{name}\PY{p}{)}\PY{p}{:}
            \PY{k}{def} \PY{n+nf}{fn}\PY{p}{(}\PY{n}{change}\PY{p}{)}\PY{p}{:}
                \PY{n}{s}\PY{o}{=}\PY{n}{shelve}\PY{o}{.}\PY{n}{open}\PY{p}{(}\PY{l+s+s1}{\PYZsq{}}\PY{l+s+s1}{.choices.shelve}\PY{l+s+s1}{\PYZsq{}}\PY{p}{)}
                \PY{n}{key}\PY{o}{=}\PY{l+s+s1}{\PYZsq{}}\PY{l+s+si}{\PYZob{}0\PYZcb{}}\PY{l+s+s1}{\PYZus{}}\PY{l+s+si}{\PYZob{}1\PYZcb{}}\PY{l+s+s1}{\PYZsq{}}\PY{o}{.}\PY{n}{format}\PY{p}{(}\PY{n}{assessmentName}\PY{p}{,}\PY{n}{name}\PY{p}{)}
                \PY{n}{s}\PY{p}{[}\PY{n}{key}\PY{p}{]}\PY{o}{=}\PY{n}{change}\PY{p}{[}\PY{l+s+s1}{\PYZsq{}}\PY{l+s+s1}{new}\PY{l+s+s1}{\PYZsq{}}\PY{p}{]}
                \PY{n}{s}\PY{o}{.}\PY{n}{close}\PY{p}{(}\PY{p}{)}
            \PY{k}{return} \PY{n}{fn}
         
        \PY{k}{class} \PY{n+nc}{myFloatBox}\PY{p}{:}
            \PY{k}{def} \PY{n+nf}{\PYZus{}\PYZus{}init\PYZus{}\PYZus{}}\PY{p}{(}\PY{n+nb+bp}{self}\PY{p}{,}\PY{n}{name}\PY{p}{,}\PY{n}{description}\PY{p}{,}\PY{n}{long\PYZus{}description}\PY{p}{)}\PY{p}{:}
                \PY{n+nb+bp}{self}\PY{o}{.}\PY{n}{name}\PY{o}{=}\PY{n}{name}
                \PY{n+nb+bp}{self}\PY{o}{.}\PY{n}{description}\PY{o}{=}\PY{n}{description}
                \PY{n+nb+bp}{self}\PY{o}{.}\PY{n}{long\PYZus{}description}\PY{o}{=}\PY{n}{long\PYZus{}description}
            \PY{k}{def} \PY{n+nf}{getWidget}\PY{p}{(}\PY{n+nb+bp}{self}\PY{p}{)}\PY{p}{:}
                \PY{n+nb+bp}{self}\PY{o}{.}\PY{n}{widgets}\PY{o}{=}\PY{p}{[} 
                      \PY{n}{widgets}\PY{o}{.}\PY{n}{FloatText}\PY{p}{(}
                      \PY{n}{description}\PY{o}{=}\PY{n+nb+bp}{self}\PY{o}{.}\PY{n}{description}\PY{p}{,}
                \PY{n}{disabled}\PY{o}{=}\PY{k+kc}{False}\PY{p}{,}
                          \PY{n}{value}\PY{o}{=}\PY{n}{get\PYZus{}last\PYZus{}value}\PY{p}{(}\PY{l+s+s1}{\PYZsq{}}\PY{l+s+si}{\PYZob{}0\PYZcb{}}\PY{l+s+s1}{\PYZus{}}\PY{l+s+si}{\PYZob{}1\PYZcb{}}\PY{l+s+s1}{\PYZsq{}}\PY{o}{.}\PY{n}{format}\PY{p}{(}\PY{n}{assessmentName}\PY{p}{,}\PY{n+nb+bp}{self}\PY{o}{.}\PY{n}{name}\PY{p}{)}\PY{p}{)}
                \PY{p}{)}\PY{p}{]}
                
                \PY{n}{txt}\PY{o}{=}\PY{n}{widgets}\PY{o}{.}\PY{n}{HTMLMath}\PY{p}{(}
                    \PY{n}{value}\PY{o}{=}\PY{n+nb+bp}{self}\PY{o}{.}\PY{n}{long\PYZus{}description}\PY{p}{,}
                    \PY{n}{placeholder}\PY{o}{=}\PY{l+s+s1}{\PYZsq{}}\PY{l+s+s1}{\PYZsq{}}\PY{p}{,}
                    \PY{n}{description}\PY{o}{=}\PY{l+s+s1}{\PYZsq{}}\PY{l+s+s1}{\PYZsq{}}\PY{p}{,}
                \PY{p}{)}
                
                \PY{n+nb+bp}{self}\PY{o}{.}\PY{n}{widget}\PY{o}{=}\PY{n}{VBox}\PY{p}{(}\PY{p}{[}\PY{n}{txt}\PY{p}{]}\PY{o}{+}\PY{n+nb+bp}{self}\PY{o}{.}\PY{n}{widgets}\PY{p}{)}
                \PY{n+nb+bp}{self}\PY{o}{.}\PY{n}{widgets}\PY{p}{[}\PY{l+m+mi}{0}\PY{p}{]}\PY{o}{.}\PY{n}{observe}\PY{p}{(}\PY{n}{make\PYZus{}value\PYZus{}change\PYZus{}fn}\PY{p}{(}\PY{n}{assessmentName}\PY{p}{,}\PY{n+nb+bp}{self}\PY{o}{.}\PY{n}{name}\PY{p}{)}\PY{p}{,} \PY{n}{names}\PY{o}{=}\PY{l+s+s1}{\PYZsq{}}\PY{l+s+s1}{value}\PY{l+s+s1}{\PYZsq{}}\PY{p}{)}
        
                \PY{k}{return} \PY{n+nb+bp}{self}\PY{o}{.}\PY{n}{widget}
\end{Verbatim}


    \hypertarget{ising-model}{%
\subsubsection{Ising model}\label{ising-model}}

The task for this assignment is to implement the Ising Model introduced
in the lecture. The structure in terms of a code skeleton provided below
needs to be followed. Otherwise the automatic tests, which allow you to
test different parts of you implementation, will not work.

We consider an Ising model, in which the interaction energy
\(E_i=E(s_i)\) of spin \(i\) is calculated from

\begin{align}
E(s_{i}) = \frac{J}{2} \sum\limits_{j} (1-s_{i} s_{j})
\end{align}

where the sum is over the 4 nearest neighbours of \(i\).

We will restrict the calculation to a 2 dimensional grid throughout,
hence 4 nearest neighbours, \(x=\pm 1\), \(y=\pm 1\). Notice that the
expression for \(E\) is different from the form considered in the
lecture.

To simplify the calculations, we will use the dimensionless interaction
energy \[
\mathcal{E}(s_\mathrm{i})  \equiv \frac{\beta}{2} \sum\limits_{j} (1-s_{i} s_{j}),
\] where \[
\beta = \frac{J}{kT},
\] in the following. Here, \(k\) is Boltzmann's constant, and \(T\) is
the temperature.

Given all \(N\) spin states, we calculate the ensenble-averaged
macroscopic magnetization \(\bar{M}\), as \begin{align}
\bar{M} = \left\langle\left|\frac{1}{N}\sum_{i=1}^N s_{i}\right|\right\rangle
\end{align}

The \(\langle\rangle\) brackets denote the ensemble average. The
parameter \(J>0\) has the dimensions of energy, and consequently
\(\beta\) is dimensionless.

Follow the numbered steps in the following cells.

    The cells below describe how to proceed, step-by-step. Begin by reading
through all steps, comparing the instructions to the Ising model
discussed in the lecture. Complete the assignment using the cells below.
Several cells allow you to test your implementation step by step.

    \hypertarget{set-up-the-regular-grid}{%
\paragraph{1. Set up the regular grid}\label{set-up-the-regular-grid}}

Set up a 2D grid in the form of a \textbf{python class}, called
\texttt{Grid}. The class should contain - the spin array - the value of
\(J\)

The spin array should be a 2D array of dimension \(L^2\), with \(L=32\).
We will address a particular spin with its 2D Cartesian coordinates
\((x,y)\), where \(x=0\cdots L-1\) and \(y=1\cdots L-1\) are the indices
in the 2D array. So, for example, spin \(s_{xy}\) refers to the spin
located at vertex \((x,y)\) of the grid.

Initialize the spins on the grid randomly, meaning each spin can be
either up, \(s_{xy}=1\), or down, \(s_{xy}=-1\), with equal probability.

When performing calculations on the grid, we will assume
\textbf{periodic boundary conditions}

** no marks **

    \hypertarget{calculate-the-energy}{%
\paragraph{2. Calculate the energy}\label{calculate-the-energy}}

Write a method, \texttt{energy}, as part of the class \texttt{Grid},
which calculates the interaction energy of a given spin, \(s_{xy}\), by
summing over its four nearest neighbours. The function should take the
grid array, \(\beta\), and the cell indices \(x\) and \(y\) as
parameters. It should \textbf{return a python tuple} containing two
dimensionless energies corresponding to the energy of the current spin
state of cell \(xy\),
\(\mathcal{E}_\mathrm{c} \equiv \mathcal{E}\left(s_{xy}^\mathrm{current}\right)\)
and the energy of the flipped spin state
\(\mathcal{E}_\mathrm{f} \equiv \mathcal{E}\left(s_{xy}^\mathrm{flipped}\right)\).

This means that for a cell with spin state \(s_{xy} = 1\), the method
should return
\(\left(\mathcal{E}\left(s_{xy} = 1\right), \mathcal{E}\left(s_{xy} = -1\right)\right)\)
and vice versa.

** Remember to account for periodic boudnary conditions on the grid.**

You can test the implementation of this method using the test cells
provded. What are the interaction energies of cells (6,6), (15,0) and
(31, 17) of the assignment grid given below (please include the answer
to this question in the PDF you hand in).

** 2 marks**

    \hypertarget{calculate-the-probability-of-flipping-a-spin}{%
\paragraph{3. Calculate the probability of flipping a
spin}\label{calculate-the-probability-of-flipping-a-spin}}

The probability that the spin at vertex \((x,y)\) is flipped depends on
the spin states of its neighbours and the value of \(\beta\) as
explained in the lecture.

Write a method \texttt{prob\_flip} which calculates the probability that
spin \(s_{xy}\) is flipped, given the (dimensionless) interaction
energies for the current state \(\mathcal{E}_\mathrm{c}\) and the
flipped state \(\mathcal{E}_\mathrm{f}\).

The probability for a flip is given by \begin{align}
\mathcal{P}_\mathrm{flip} = 
\begin{cases}
\exp\left(-\left[\mathcal{E}_\mathrm{f} - \mathcal{E}_\mathrm{c}\right]\right) & \text{if } \mathcal{E}_\mathrm{f} > \mathcal{E}_\mathrm{c}, \\
1 & \text{if } \mathcal{E}_\mathrm{f} \leq \mathcal{E}_\mathrm{c}.
\end{cases}
\end{align}

You can test the implementation of this method using the test cells
provided. What are the probabilities for cells (12, 12), (18,0) and (31,
12) of the assignment grid given below (please include the answer to
this question in the PDF you hand in)?

** 2 marks **

    \hypertarget{calculate-the-macroscopic-magnetisation-m}{%
\paragraph{\texorpdfstring{4. Calculate the macroscopic magnetisation,
\(M\)}{4. Calculate the macroscopic magnetisation, M}}\label{calculate-the-macroscopic-magnetisation-m}}

Write a method which calculates the current macroscopic magnetisation of
a given grid, and add it to the \texttt{Grid} class. The function should
take the grid-array as a parameter and return the mean, macroscopic
magnetisation,

\[ M=\frac{1}{N}\sum_{i=1}^N s_i\,.\]

You can test the implementation of this method using the test cells
provded. Calculate the magnetisation of the assignment grid. State the
answer to 3 significant digits on the PDF file you hand in.

\textbf{2 marks}

    \hypertarget{red-black-sweep}{%
\paragraph{5. Red-black sweep}\label{red-black-sweep}}

Write a method to sweep over all spins in turn, first in \(x\) (say) and
then in \(y\) in a \textbf{red-black pattern}. Red-black means, first
loop over all the red cells on a red-black chessboard pattern (looping
over them \(x\) first, then \(y\)). Once all the red cells are done,
then update all the black cells. Add this method to the \texttt{Grid}
class.

For each spin in turn, flip the spin or not following the criterion
discussed in the lecture. This means that the spin in each cell in turn
should be flipped with a probability \(\mathcal{P}_\mathrm{flip}\) (as
discussed in step \textbf{3}).

You can use the methods implemented in step \textbf{2} and \textbf{3}.

** no marks **

    \hypertarget{thermalisation-and-magnetisation-measurement}{%
\paragraph{6. Thermalisation and magnetisation
measurement}\label{thermalisation-and-magnetisation-measurement}}

Starting from a random configuration, the system needs to be evolved
over a number of full red-black sweeps in order to reach thermal
equilibrium. This \emph{thermalization} is part of the method you
develop in this step.

Write a method that starts by sweeping the grid \(N_\mathrm{therm}\)
times to allow for the system to thermalize before you carry out any
measurements.

Next, the method should perform a further \(N_\mathrm{measure}\) sweeps,
while in addition computing and recording the value of \(M\) after every
sweep. Use the method you developed in step \textbf{4} and the sweep
implementaton of step \textbf{5}.

\(N_\mathrm{therm}\) and \(N_\mathrm{measure}\) are input parameters of
the method. The method should return a numpy array of length
\(N_\mathrm{measure}\), containing the magentisations measured after
each measurement sweep.

Add this method to the \texttt{Grid} class.

\textbf{no marks}

    \hypertarget{thermalisation}{%
\paragraph{7. Thermalisation}\label{thermalisation}}

Plot the magnetisation over time for 1000 full mesh sweeps for
\(\beta = 0.1, 0.8\) and \(1.6\) (include the thermalisation period in
the plot). Include this plot in the PDF you hand in. Save the plot to a
file `Thermalization.pdf'

\textbf{4 marks}

    \hypertarget{ensemble-averaged-magnetisation-bar-m-as-a-function-of-beta}{%
\paragraph{\texorpdfstring{8. Ensemble-averaged magnetisation,
\(\bar M\), as a function of
\(\beta\)}{8. Ensemble-averaged magnetisation, \textbackslash{}bar M, as a function of \textbackslash{}beta}}\label{ensemble-averaged-magnetisation-bar-m-as-a-function-of-beta}}

Once the system has thermalized (a good choice is
\(N_\mathrm{therm} =400\) thermalisation sweeps), measure the
time-averaged magnetisation over 1000 sweeps. From this, estimate the
ensemble-averaged magnetisation, \(\bar M\). Plot \(\bar M\) as a
function of \(1/\beta\) for
\(\beta = 1.6, 1.3, 1.1, 1.0, 0.9, 0.8, 0.7, 0.6, 0.4, 0.1\). Include
this plot in the PDF you hand in. Save the plot to a file
`Magnetisation.pdf'.

\textbf{4 marks}

    \hypertarget{critical-temperature}{%
\paragraph{9. Critical temperature}\label{critical-temperature}}

The critical temperature, \(T_c\), in the mean field approximation, is
given by \[
T_c=\frac{h J}{2k_B},
\] where h is the number of nearest-neighbours, as discussed in the
lecture. Use this to calculate the critical value \(\beta_c\) which
corresponds to the critical temperature and mark it in the plot produced
in step \textbf{8} (If you could not produce the plot in step \textbf{8}
you may also mention the numerical value of \(\beta_c\) on the solution
you hand in).

\textbf{2 marks}

    \hypertarget{mean-field-approximation}{%
\paragraph{10. Mean field
approximation}\label{mean-field-approximation}}

The ensemble-averaged value of the mean magnitization in the mean field
approximation, is the solution of \[
\bar M - \tanh\left( \frac{T_c\bar M}{T} \right) = 0\,,
\] as discussed in the lecture. This equation can not be solved
analytically for \(\bar M\), for given \(T_c/T\).

Rewrite the equation in terms of \(\beta\) using the relation between
\(T\) and \(\beta\) derived before, and solve the resulting formula
numerically for
\(\beta = 1.6, 1.3, 1.1, 1.0, 0.9, 0.8, 0.7, 0.6, 0.4, 0.1\). You may
want to use the root finding methods implemented in previous exercises.
Redo the plot of step \textbf{8}, but on top of the numerical result,
over plot the mean field approximation. Add a legend to the plot which
allows to distinguish the solution obtained in step \textbf{8} from the
mean field approximation.

Save the plot as `MeanField.pdf'

\textbf{4 marks}

    \hypertarget{solution}{%
\subsection{Solution}\label{solution}}

Use the cells below to complete the assignment. Use the test cells
provided to make sure you are on the right track.

    \begin{Verbatim}[commandchars=\\\{\}]
{\color{incolor}In [{\color{incolor}155}]:} \PY{o}{\PYZpc{}}\PY{k}{matplotlib} inline
          \PY{k+kn}{import} \PY{n+nn}{numpy} \PY{k}{as} \PY{n+nn}{np}
          \PY{k+kn}{import} \PY{n+nn}{matplotlib}\PY{n+nn}{.}\PY{n+nn}{pyplot} \PY{k}{as} \PY{n+nn}{plt} 
          \PY{k+kn}{import} \PY{n+nn}{sys}
          \PY{k+kn}{import} \PY{n+nn}{os}
          \PY{n}{sys}\PY{o}{.}\PY{n}{path}\PY{o}{.}\PY{n}{append}\PY{p}{(}\PY{n}{os}\PY{o}{.}\PY{n}{getcwd}\PY{p}{(}\PY{p}{)}\PY{p}{)}
          \PY{k+kn}{from} \PY{n+nn}{scipy}\PY{n+nn}{.}\PY{n+nn}{interpolate} \PY{k}{import} \PY{n}{CubicSpline}
          \PY{k+kn}{import} \PY{n+nn}{pickle}
\end{Verbatim}


    \begin{Verbatim}[commandchars=\\\{\}]
{\color{incolor}In [{\color{incolor}156}]:} \PY{c+c1}{\PYZsh{} This cell is hidden from the student\PYZsq{}s notebook. It generates the buttons used in the answers.}
          \PY{k+kn}{from} \PY{n+nn}{ipywidgets} \PY{k}{import} \PY{n}{widgets}\PY{p}{,} \PY{n}{interact}\PY{p}{,} \PY{n}{interactive}\PY{p}{,} \PY{n}{fixed}
          \PY{k+kn}{from} \PY{n+nn}{ipywidgets} \PY{k}{import} \PY{n}{Button}\PY{p}{,} \PY{n}{HBox}\PY{p}{,} \PY{n}{VBox}
          \PY{k+kn}{import} \PY{n+nn}{shelve}
          \PY{n}{assessmentName}\PY{o}{=}\PY{l+s+s2}{\PYZdq{}}\PY{l+s+s2}{test}\PY{l+s+s2}{\PYZdq{}}\PY{p}{;}
          \PY{k+kn}{import} \PY{n+nn}{os}
          
          \PY{k}{def} \PY{n+nf}{get\PYZus{}last\PYZus{}value}\PY{p}{(}\PY{n}{key}\PY{p}{)}\PY{p}{:}
              \PY{k}{if} \PY{n}{os}\PY{o}{.}\PY{n}{path}\PY{o}{.}\PY{n}{isfile}\PY{p}{(}\PY{l+s+s1}{\PYZsq{}}\PY{l+s+s1}{.choices.shelve}\PY{l+s+s1}{\PYZsq{}}\PY{p}{)} \PY{o+ow}{or} \PY{n}{os}\PY{o}{.}\PY{n}{path}\PY{o}{.}\PY{n}{isfile}\PY{p}{(}\PY{l+s+s1}{\PYZsq{}}\PY{l+s+s1}{.choices.shelve.dir}\PY{l+s+s1}{\PYZsq{}}\PY{p}{)}\PY{p}{:}
                  \PY{n}{s}\PY{o}{=}\PY{n}{shelve}\PY{o}{.}\PY{n}{open}\PY{p}{(}\PY{l+s+s1}{\PYZsq{}}\PY{l+s+s1}{.choices.shelve}\PY{l+s+s1}{\PYZsq{}}\PY{p}{)}
                  \PY{k}{return} \PY{n}{s}\PY{o}{.}\PY{n}{get}\PY{p}{(}\PY{n}{key}\PY{p}{,}\PY{k+kc}{None}\PY{p}{)}
              \PY{k}{return} \PY{k+kc}{None}
          
          
          \PY{k}{class} \PY{n+nc}{myRadioButton}\PY{p}{:}
              \PY{k}{def} \PY{n+nf}{\PYZus{}\PYZus{}init\PYZus{}\PYZus{}}\PY{p}{(}\PY{n+nb+bp}{self}\PY{p}{,}\PY{n}{name}\PY{p}{,}\PY{n}{description}\PY{p}{,}\PY{n}{options}\PY{p}{)}\PY{p}{:}
                  \PY{n+nb+bp}{self}\PY{o}{.}\PY{n}{name}\PY{o}{=}\PY{n}{name}
                  \PY{n+nb+bp}{self}\PY{o}{.}\PY{n}{options}\PY{o}{=}\PY{n}{options}
                  \PY{n+nb+bp}{self}\PY{o}{.}\PY{n}{description}\PY{o}{=}\PY{n}{description}
              \PY{k}{def} \PY{n+nf}{getWidget}\PY{p}{(}\PY{n+nb+bp}{self}\PY{p}{)}\PY{p}{:}
                  \PY{k}{def} \PY{n+nf}{on\PYZus{}value\PYZus{}change}\PY{p}{(}\PY{n}{change}\PY{p}{)}\PY{p}{:}
                      \PY{n}{s}\PY{o}{=}\PY{n}{shelve}\PY{o}{.}\PY{n}{open}\PY{p}{(}\PY{l+s+s1}{\PYZsq{}}\PY{l+s+s1}{.choices.shelve}\PY{l+s+s1}{\PYZsq{}}\PY{p}{)}
                      \PY{n}{key}\PY{o}{=}\PY{n+nb+bp}{self}\PY{o}{.}\PY{n}{getKey}\PY{p}{(}\PY{p}{)}
                      \PY{n}{s}\PY{p}{[}\PY{n}{key}\PY{p}{]}\PY{o}{=}\PY{n}{change}\PY{p}{[}\PY{l+s+s1}{\PYZsq{}}\PY{l+s+s1}{new}\PY{l+s+s1}{\PYZsq{}}\PY{p}{]}
                      \PY{n}{s}\PY{o}{.}\PY{n}{close}\PY{p}{(}\PY{p}{)}
          
                  \PY{n+nb+bp}{self}\PY{o}{.}\PY{n}{widget}\PY{o}{=}\PY{n}{widgets}\PY{o}{.}\PY{n}{RadioButtons}\PY{p}{(}
                      \PY{n}{options}\PY{o}{=}\PY{n+nb+bp}{self}\PY{o}{.}\PY{n}{options}\PY{p}{,}
                      \PY{n}{value}\PY{o}{=}\PY{n}{get\PYZus{}last\PYZus{}value}\PY{p}{(}\PY{n+nb+bp}{self}\PY{o}{.}\PY{n}{getKey}\PY{p}{(}\PY{p}{)}\PY{p}{)}\PY{p}{,}
                      \PY{n}{description}\PY{o}{=}\PY{n+nb+bp}{self}\PY{o}{.}\PY{n}{description}\PY{p}{,}
                      \PY{n}{disabled}\PY{o}{=}\PY{k+kc}{False}
                  \PY{p}{)}
                  \PY{n+nb+bp}{self}\PY{o}{.}\PY{n}{widget}\PY{o}{.}\PY{n}{observe}\PY{p}{(}\PY{n}{on\PYZus{}value\PYZus{}change}\PY{p}{,} \PY{n}{names}\PY{o}{=}\PY{l+s+s1}{\PYZsq{}}\PY{l+s+s1}{value}\PY{l+s+s1}{\PYZsq{}}\PY{p}{)}
          
                  \PY{k}{return} \PY{n+nb+bp}{self}\PY{o}{.}\PY{n}{widget}
              \PY{k}{def} \PY{n+nf}{getKey}\PY{p}{(}\PY{n+nb+bp}{self}\PY{p}{)}\PY{p}{:}
                  \PY{k}{return} \PY{l+s+s1}{\PYZsq{}}\PY{l+s+si}{\PYZob{}0\PYZcb{}}\PY{l+s+s1}{\PYZus{}}\PY{l+s+si}{\PYZob{}1\PYZcb{}}\PY{l+s+s1}{\PYZsq{}}\PY{o}{.}\PY{n}{format}\PY{p}{(}\PY{n}{assessmentName}\PY{p}{,}\PY{n+nb+bp}{self}\PY{o}{.}\PY{n}{name}\PY{p}{)}
                  
                  
          \PY{k}{def} \PY{n+nf}{on\PYZus{}value\PYZus{}change}\PY{p}{(}\PY{n}{change}\PY{p}{)}\PY{p}{:}
                          \PY{n}{s}\PY{o}{=}\PY{n}{shelve}\PY{o}{.}\PY{n}{open}\PY{p}{(}\PY{l+s+s1}{\PYZsq{}}\PY{l+s+s1}{.choices.shelve}\PY{l+s+s1}{\PYZsq{}}\PY{p}{)}
                          \PY{n}{key}\PY{o}{=}\PY{l+s+s1}{\PYZsq{}}\PY{l+s+si}{\PYZob{}0\PYZcb{}}\PY{l+s+s1}{\PYZus{}}\PY{l+s+si}{\PYZob{}1\PYZcb{}}\PY{l+s+s1}{\PYZus{}}\PY{l+s+si}{\PYZob{}2\PYZcb{}}\PY{l+s+s1}{\PYZsq{}}\PY{o}{.}\PY{n}{format}\PY{p}{(}\PY{n}{assessmentName}\PY{p}{,}\PY{n+nb+bp}{self}\PY{o}{.}\PY{n}{name}\PY{p}{,}\PY{n}{i}\PY{p}{)}
                          \PY{n}{s}\PY{p}{[}\PY{n}{key}\PY{p}{]}\PY{o}{=}\PY{n}{change}\PY{p}{[}\PY{l+s+s1}{\PYZsq{}}\PY{l+s+s1}{new}\PY{l+s+s1}{\PYZsq{}}\PY{p}{]}
                          \PY{n}{s}\PY{o}{.}\PY{n}{close}\PY{p}{(}\PY{p}{)}
          
          \PY{k}{def} \PY{n+nf}{make\PYZus{}value\PYZus{}change\PYZus{}fn}\PY{p}{(}\PY{n}{assessmentName}\PY{p}{,}\PY{n}{name}\PY{p}{,}\PY{n}{i}\PY{p}{)}\PY{p}{:}
                  \PY{k}{def} \PY{n+nf}{fn}\PY{p}{(}\PY{n}{change}\PY{p}{)}\PY{p}{:}
                      \PY{n}{s}\PY{o}{=}\PY{n}{shelve}\PY{o}{.}\PY{n}{open}\PY{p}{(}\PY{l+s+s1}{\PYZsq{}}\PY{l+s+s1}{.choices.shelve}\PY{l+s+s1}{\PYZsq{}}\PY{p}{)}
                      \PY{n}{key}\PY{o}{=}\PY{l+s+s1}{\PYZsq{}}\PY{l+s+si}{\PYZob{}0\PYZcb{}}\PY{l+s+s1}{\PYZus{}}\PY{l+s+si}{\PYZob{}1\PYZcb{}}\PY{l+s+s1}{\PYZus{}}\PY{l+s+si}{\PYZob{}2\PYZcb{}}\PY{l+s+s1}{\PYZsq{}}\PY{o}{.}\PY{n}{format}\PY{p}{(}\PY{n}{assessmentName}\PY{p}{,}\PY{n}{name}\PY{p}{,}\PY{n}{i}\PY{p}{)}
                      \PY{n}{s}\PY{p}{[}\PY{n}{key}\PY{p}{]}\PY{o}{=}\PY{n}{change}\PY{p}{[}\PY{l+s+s1}{\PYZsq{}}\PY{l+s+s1}{new}\PY{l+s+s1}{\PYZsq{}}\PY{p}{]}
                      \PY{n}{s}\PY{o}{.}\PY{n}{close}\PY{p}{(}\PY{p}{)}
                  \PY{k}{return} \PY{n}{fn}
          
          \PY{k}{class} \PY{n+nc}{myCheckBoxSet}\PY{p}{:}
              \PY{k}{def} \PY{n+nf}{\PYZus{}\PYZus{}init\PYZus{}\PYZus{}}\PY{p}{(}\PY{n+nb+bp}{self}\PY{p}{,}\PY{n}{name}\PY{p}{,}\PY{n}{description}\PY{p}{,}\PY{n}{options}\PY{p}{)}\PY{p}{:}
                  \PY{n+nb+bp}{self}\PY{o}{.}\PY{n}{name}\PY{o}{=}\PY{n}{name}
                  \PY{n+nb+bp}{self}\PY{o}{.}\PY{n}{options}\PY{o}{=}\PY{n}{options}
                  \PY{n+nb+bp}{self}\PY{o}{.}\PY{n}{description}\PY{o}{=}\PY{n}{description}
              \PY{k}{def} \PY{n+nf}{getWidget}\PY{p}{(}\PY{n+nb+bp}{self}\PY{p}{)}\PY{p}{:}
                  \PY{n}{keys}\PY{o}{=}\PY{p}{[}\PY{l+s+s1}{\PYZsq{}}\PY{l+s+si}{\PYZob{}0\PYZcb{}}\PY{l+s+s1}{\PYZus{}}\PY{l+s+si}{\PYZob{}1\PYZcb{}}\PY{l+s+s1}{\PYZus{}}\PY{l+s+si}{\PYZob{}2\PYZcb{}}\PY{l+s+s1}{\PYZsq{}}\PY{o}{.}\PY{n}{format}\PY{p}{(}\PY{n}{assessmentName}\PY{p}{,}\PY{n+nb+bp}{self}\PY{o}{.}\PY{n}{name}\PY{p}{,}\PY{n}{i}\PY{p}{)} \PY{k}{for} \PY{n}{i} \PY{o+ow}{in} \PY{n+nb}{range}\PY{p}{(}\PY{n+nb}{len}\PY{p}{(}\PY{n+nb+bp}{self}\PY{o}{.}\PY{n}{options}\PY{p}{)}\PY{p}{)}\PY{p}{]}    
                  \PY{n+nb+bp}{self}\PY{o}{.}\PY{n}{widgets}\PY{o}{=}\PY{p}{[} \PY{n}{widgets}\PY{o}{.}\PY{n}{Checkbox}\PY{p}{(}\PY{n}{value}\PY{o}{=}\PY{n}{get\PYZus{}last\PYZus{}value}\PY{p}{(}\PY{n}{key}\PY{p}{)}\PY{p}{,}
              \PY{n}{description}\PY{o}{=}\PY{n}{o}\PY{p}{,}
              \PY{n}{disabled}\PY{o}{=}\PY{k+kc}{False}
                  \PY{p}{)} \PY{k}{for} \PY{n}{key}\PY{p}{,}\PY{n}{o} \PY{o+ow}{in} \PY{n+nb}{zip}\PY{p}{(}\PY{n}{keys}\PY{p}{,}\PY{n+nb+bp}{self}\PY{o}{.}\PY{n}{options}\PY{p}{)}\PY{p}{]}
                  
                  \PY{n}{txt}\PY{o}{=}\PY{n}{widgets}\PY{o}{.}\PY{n}{HTMLMath}\PY{p}{(}
                      \PY{n}{value}\PY{o}{=}\PY{n+nb+bp}{self}\PY{o}{.}\PY{n}{description}\PY{p}{,}
                      \PY{n}{placeholder}\PY{o}{=}\PY{l+s+s1}{\PYZsq{}}\PY{l+s+s1}{\PYZsq{}}\PY{p}{,}
                      \PY{n}{description}\PY{o}{=}\PY{l+s+s1}{\PYZsq{}}\PY{l+s+s1}{\PYZsq{}}\PY{p}{,}
                  \PY{p}{)}
          
                  
                  \PY{n+nb+bp}{self}\PY{o}{.}\PY{n}{widget}\PY{o}{=}\PY{n}{VBox}\PY{p}{(}\PY{p}{[}\PY{n}{txt}\PY{p}{]}\PY{o}{+}\PY{n+nb+bp}{self}\PY{o}{.}\PY{n}{widgets}\PY{p}{)}
                  \PY{k}{for} \PY{n}{i}\PY{p}{,}\PY{n}{w} \PY{o+ow}{in} \PY{n+nb}{enumerate}\PY{p}{(}\PY{n+nb+bp}{self}\PY{o}{.}\PY{n}{widgets}\PY{p}{)}\PY{p}{:}
                      \PY{n}{w}\PY{o}{.}\PY{n}{observe}\PY{p}{(}\PY{n}{make\PYZus{}value\PYZus{}change\PYZus{}fn}\PY{p}{(}\PY{n}{assessmentName}\PY{p}{,}\PY{n+nb+bp}{self}\PY{o}{.}\PY{n}{name}\PY{p}{,}\PY{n}{i}\PY{p}{)}\PY{p}{,} \PY{n}{names}\PY{o}{=}\PY{l+s+s1}{\PYZsq{}}\PY{l+s+s1}{value}\PY{l+s+s1}{\PYZsq{}}\PY{p}{)}
          
                  \PY{k}{return} \PY{n+nb+bp}{self}\PY{o}{.}\PY{n}{widget}
          \PY{k+kn}{import} \PY{n+nn}{mywidgets}    
\end{Verbatim}


    \begin{Verbatim}[commandchars=\\\{\}]
{\color{incolor}In [{\color{incolor}120}]:} \PY{n+nb}{print}\PY{p}{(}\PY{n}{np}\PY{o}{.}\PY{n}{random}\PY{o}{.}\PY{n}{random}\PY{p}{(}\PY{p}{)}\PY{p}{)}
\end{Verbatim}


    \begin{Verbatim}[commandchars=\\\{\}]
0.09294653669464614

    \end{Verbatim}

    \begin{Verbatim}[commandchars=\\\{\}]
{\color{incolor}In [{\color{incolor}161}]:} \PY{k}{class} \PY{n+nc}{Grid}\PY{p}{:}
              \PY{k}{def} \PY{n+nf}{\PYZus{}\PYZus{}init\PYZus{}\PYZus{}}\PY{p}{(}\PY{n+nb+bp}{self}\PY{p}{,} \PY{n}{size}\PY{p}{,} \PY{n}{beta}\PY{p}{)}\PY{p}{:}
                  \PY{l+s+sd}{\PYZsq{}\PYZsq{}\PYZsq{}This function initialises the grid, i.e. it sets the }
          \PY{l+s+sd}{        grid size, the value of beta and initialises the cells of the }
          \PY{l+s+sd}{        grid with randomly chosen \PYZsq{}plus\PYZsq{} (1) or \PYZsq{}minus\PYZsq{} (\PYZhy{}1) states.\PYZsq{}\PYZsq{}\PYZsq{}}
                  \PY{c+c1}{\PYZsh{} set self.size, self.beta, and self.cells}
                  \PY{c+c1}{\PYZsh{} self.cells is a 2D array, so that self.cells[i,j]=+1 or \PYZhy{}1, the spin in grid location (i,j)}
                  \PY{c+c1}{\PYZsh{} YOUR CODE HERE}
                  \PY{n+nb+bp}{self}\PY{o}{.}\PY{n}{size} \PY{o}{=} \PY{n}{size}
                  \PY{n+nb+bp}{self}\PY{o}{.}\PY{n}{beta} \PY{o}{=} \PY{n}{beta}
                  \PY{n+nb+bp}{self}\PY{o}{.}\PY{n}{cells} \PY{o}{=} \PY{p}{[}\PY{p}{]}
                  \PY{k}{for} \PY{n}{i} \PY{o+ow}{in} \PY{n+nb}{range}\PY{p}{(}\PY{n}{size}\PY{p}{)}\PY{p}{:}
                      \PY{n}{tmp} \PY{o}{=} \PY{p}{[}\PY{p}{]}
                      \PY{k}{for} \PY{n}{j} \PY{o+ow}{in} \PY{n+nb}{range}\PY{p}{(}\PY{n}{size}\PY{p}{)}\PY{p}{:}
                          \PY{k}{if} \PY{p}{(}\PY{n}{np}\PY{o}{.}\PY{n}{random}\PY{o}{.}\PY{n}{random}\PY{p}{(}\PY{p}{)} \PY{o}{\PYZgt{}} \PY{l+m+mf}{0.5}\PY{p}{)}\PY{p}{:}
                              \PY{n}{tmp}\PY{o}{.}\PY{n}{append}\PY{p}{(}\PY{l+m+mi}{1}\PY{p}{)}
                          \PY{k}{else}\PY{p}{:}
                              \PY{n}{tmp}\PY{o}{.}\PY{n}{append}\PY{p}{(}\PY{o}{\PYZhy{}}\PY{l+m+mi}{1}\PY{p}{)}
                      \PY{n+nb+bp}{self}\PY{o}{.}\PY{n}{cells}\PY{o}{.}\PY{n}{append}\PY{p}{(}\PY{n}{tmp}\PY{p}{)}
          \PY{c+c1}{\PYZsh{}         raise NotImplementedError()}
                          
              \PY{k}{def} \PY{n+nf}{energy}\PY{p}{(}\PY{n+nb+bp}{self}\PY{p}{,} \PY{n}{i}\PY{p}{,} \PY{n}{j}\PY{p}{,} \PY{n}{beta}\PY{p}{,} \PY{n}{grid}\PY{p}{)}\PY{p}{:}
                  \PY{l+s+sd}{\PYZsq{}\PYZsq{}\PYZsq{}This function calculates the energies \PYZsq{}e\PYZus{}plus\PYZsq{} and \PYZsq{}e\PYZus{}minus\PYZsq{} }
          \PY{l+s+sd}{        corresponding to the two possible states \PYZsq{}plus\PYZsq{} and \PYZsq{}minus\PYZsq{} for the spin at location (i, j)}
          \PY{l+s+sd}{        of a given grid with a given value of beta.}
          \PY{l+s+sd}{        returns: the two energy states \PYZsq{}e\PYZus{}current\PYZsq{} and \PYZsq{}e\PYZus{}flip\PYZsq{} as a tuple. }
          \PY{l+s+sd}{        \PYZsq{}e\PYZus{}current\PYZsq{} is the energy of the spin (i,j) in its current spin state}
          \PY{l+s+sd}{        \PYZsq{}e\PYZus{}flip\PYZsq{} is the energy of the spin (i,j) if you were to flip its spin}
          \PY{l+s+sd}{        \PYZsq{}\PYZsq{}\PYZsq{}}
                  \PY{c+c1}{\PYZsh{} YOUR CODE HERE}
                  \PY{n}{e\PYZus{}current} \PY{o}{=} \PY{l+m+mi}{0}
                  \PY{k}{if} \PY{n}{i} \PY{o}{+} \PY{l+m+mi}{1} \PY{o}{\PYZgt{}}\PY{o}{=} \PY{n+nb}{len}\PY{p}{(}\PY{n}{grid}\PY{p}{)}\PY{p}{:}
                      \PY{n}{e\PYZus{}current} \PY{o}{+}\PY{o}{=} \PY{l+m+mi}{1} \PY{o}{\PYZhy{}} \PY{n}{grid}\PY{p}{[}\PY{l+m+mi}{0}\PY{p}{]}\PY{p}{[}\PY{n}{j}\PY{p}{]} \PY{o}{*} \PY{n}{grid}\PY{p}{[}\PY{n}{i}\PY{p}{]}\PY{p}{[}\PY{n}{j}\PY{p}{]}
                      \PY{n}{e\PYZus{}current} \PY{o}{+}\PY{o}{=} \PY{l+m+mi}{1} \PY{o}{\PYZhy{}} \PY{n}{grid}\PY{p}{[}\PY{n}{i} \PY{o}{\PYZhy{}} \PY{l+m+mi}{1}\PY{p}{]}\PY{p}{[}\PY{n}{j}\PY{p}{]} \PY{o}{*} \PY{n}{grid}\PY{p}{[}\PY{n}{i}\PY{p}{]}\PY{p}{[}\PY{n}{j}\PY{p}{]}
                  \PY{k}{else}\PY{p}{:}
                      \PY{n}{e\PYZus{}current} \PY{o}{+}\PY{o}{=} \PY{l+m+mi}{1} \PY{o}{\PYZhy{}} \PY{n}{grid}\PY{p}{[}\PY{n}{i} \PY{o}{\PYZhy{}} \PY{l+m+mi}{1}\PY{p}{]}\PY{p}{[}\PY{n}{j}\PY{p}{]} \PY{o}{*} \PY{n}{grid}\PY{p}{[}\PY{n}{i}\PY{p}{]}\PY{p}{[}\PY{n}{j}\PY{p}{]}
                      \PY{n}{e\PYZus{}current} \PY{o}{+}\PY{o}{=} \PY{l+m+mi}{1} \PY{o}{\PYZhy{}} \PY{n}{grid}\PY{p}{[}\PY{n}{i} \PY{o}{+} \PY{l+m+mi}{1}\PY{p}{]}\PY{p}{[}\PY{n}{j}\PY{p}{]} \PY{o}{*} \PY{n}{grid}\PY{p}{[}\PY{n}{i}\PY{p}{]}\PY{p}{[}\PY{n}{j}\PY{p}{]}
                      
                  \PY{k}{if} \PY{n}{j} \PY{o}{+} \PY{l+m+mi}{1} \PY{o}{\PYZgt{}}\PY{o}{=} \PY{n+nb}{len}\PY{p}{(}\PY{n}{grid}\PY{p}{[}\PY{l+m+mi}{0}\PY{p}{]}\PY{p}{)}\PY{p}{:}
                      \PY{n}{e\PYZus{}current} \PY{o}{+}\PY{o}{=} \PY{l+m+mi}{1} \PY{o}{\PYZhy{}} \PY{n}{grid}\PY{p}{[}\PY{n}{i}\PY{p}{]}\PY{p}{[}\PY{l+m+mi}{0}\PY{p}{]} \PY{o}{*} \PY{n}{grid}\PY{p}{[}\PY{n}{i}\PY{p}{]}\PY{p}{[}\PY{n}{j}\PY{p}{]}
                      \PY{n}{e\PYZus{}current} \PY{o}{+}\PY{o}{=} \PY{l+m+mi}{1} \PY{o}{\PYZhy{}} \PY{n}{grid}\PY{p}{[}\PY{n}{i}\PY{p}{]}\PY{p}{[}\PY{n}{j} \PY{o}{\PYZhy{}} \PY{l+m+mi}{1}\PY{p}{]} \PY{o}{*} \PY{n}{grid}\PY{p}{[}\PY{n}{i}\PY{p}{]}\PY{p}{[}\PY{n}{j}\PY{p}{]}
                  \PY{k}{else}\PY{p}{:}
                      \PY{n}{e\PYZus{}current} \PY{o}{+}\PY{o}{=} \PY{l+m+mi}{1} \PY{o}{\PYZhy{}} \PY{n}{grid}\PY{p}{[}\PY{n}{i}\PY{p}{]}\PY{p}{[}\PY{n}{j} \PY{o}{+} \PY{l+m+mi}{1}\PY{p}{]} \PY{o}{*} \PY{n}{grid}\PY{p}{[}\PY{n}{i}\PY{p}{]}\PY{p}{[}\PY{n}{j}\PY{p}{]}
                      \PY{n}{e\PYZus{}current} \PY{o}{+}\PY{o}{=} \PY{l+m+mi}{1} \PY{o}{\PYZhy{}} \PY{n}{grid}\PY{p}{[}\PY{n}{i}\PY{p}{]}\PY{p}{[}\PY{n}{j} \PY{o}{\PYZhy{}} \PY{l+m+mi}{1}\PY{p}{]} \PY{o}{*} \PY{n}{grid}\PY{p}{[}\PY{n}{i}\PY{p}{]}\PY{p}{[}\PY{n}{j}\PY{p}{]}
                  \PY{n}{e\PYZus{}current} \PY{o}{*}\PY{o}{=} \PY{n}{beta} \PY{o}{/} \PY{l+m+mi}{2}
                  \PY{n}{e\PYZus{}flip} \PY{o}{=} \PY{l+m+mi}{4} \PY{o}{*} \PY{n}{beta} \PY{o}{\PYZhy{}} \PY{n}{e\PYZus{}current}
                  \PY{k}{return} \PY{n}{e\PYZus{}current}\PY{p}{,} \PY{n}{e\PYZus{}flip}
              
              \PY{k}{def} \PY{n+nf}{prob\PYZus{}flip}\PY{p}{(}\PY{n+nb+bp}{self}\PY{p}{,} \PY{n}{e\PYZus{}current}\PY{p}{,} \PY{n}{e\PYZus{}flip}\PY{p}{)}\PY{p}{:}
                  \PY{l+s+sd}{\PYZsq{}\PYZsq{}\PYZsq{}This function calculates the probability of a spin flip }
          \PY{l+s+sd}{        for a given spin, given the energies e\PYZus{}current and e\PYZus{}flip }
          \PY{l+s+sd}{        of the current and the flipped state for the cell.}
          \PY{l+s+sd}{        returns: the probability for the flip\PYZsq{}\PYZsq{}\PYZsq{}}
                  \PY{c+c1}{\PYZsh{} YOUR CODE HERE}
          \PY{c+c1}{\PYZsh{}         raise NotImplementedError()}
                  \PY{n}{probability\PYZus{}for\PYZus{}flip} \PY{o}{=} \PY{l+m+mi}{0}
                  \PY{k}{if} \PY{n}{e\PYZus{}current} \PY{o}{\PYZgt{}} \PY{n}{e\PYZus{}flip}\PY{p}{:}
                      \PY{n}{probability\PYZus{}for\PYZus{}flip} \PY{o}{=} \PY{l+m+mi}{1}
                  \PY{k}{else}\PY{p}{:}
                      \PY{n}{probability\PYZus{}for\PYZus{}flip} \PY{o}{=} \PY{n}{np}\PY{o}{.}\PY{n}{exp}\PY{p}{(}\PY{o}{\PYZhy{}}\PY{l+m+mi}{1} \PY{o}{*} \PY{n+nb}{abs}\PY{p}{(}\PY{n}{e\PYZus{}flip} \PY{o}{\PYZhy{}} \PY{n}{e\PYZus{}current}\PY{p}{)}\PY{p}{)}
                  \PY{k}{return} \PY{n}{probability\PYZus{}for\PYZus{}flip}
              
              \PY{k}{def} \PY{n+nf}{sweep}\PY{p}{(}\PY{n+nb+bp}{self}\PY{p}{)}\PY{p}{:}
                  \PY{l+s+sd}{\PYZsq{}\PYZsq{}\PYZsq{}This function carries out a single red\PYZhy{}black sweep. }
          \PY{l+s+sd}{        returns: nothing. For each spin in turn, it compute the probability for }
          \PY{l+s+sd}{        flipping the spin, using the prob\PYZus{}flip function. Comparing the probablity,}
          \PY{l+s+sd}{        it draws a random number to decide whether or not the flip the spin \PYZsq{}\PYZsq{}\PYZsq{}}
                  \PY{c+c1}{\PYZsh{} YOUR CODE HERE}
                  \PY{n}{new\PYZus{}grid} \PY{o}{=} \PY{n+nb+bp}{self}\PY{o}{.}\PY{n}{cells}\PY{o}{.}\PY{n}{copy}\PY{p}{(}\PY{p}{)}
                  \PY{k}{for} \PY{n}{i} \PY{o+ow}{in} \PY{n+nb}{range}\PY{p}{(}\PY{n+nb}{len}\PY{p}{(}\PY{n}{new\PYZus{}grid}\PY{p}{)}\PY{p}{)}\PY{p}{:}
                      \PY{k}{for} \PY{n}{j} \PY{o+ow}{in} \PY{n+nb}{range}\PY{p}{(}\PY{n+nb}{len}\PY{p}{(}\PY{n}{new\PYZus{}grid}\PY{p}{[}\PY{l+m+mi}{0}\PY{p}{]}\PY{p}{)}\PY{p}{)}\PY{p}{:}
                          \PY{n}{e\PYZus{}current}\PY{p}{,} \PY{n}{e\PYZus{}flip} \PY{o}{=} \PY{n+nb+bp}{self}\PY{o}{.}\PY{n}{energy}\PY{p}{(}\PY{n}{i}\PY{p}{,} \PY{n}{j}\PY{p}{,} \PY{n+nb+bp}{self}\PY{o}{.}\PY{n}{beta}\PY{p}{,} \PY{n+nb+bp}{self}\PY{o}{.}\PY{n}{cells}\PY{p}{)}
                          \PY{n}{probability\PYZus{}for\PYZus{}flip} \PY{o}{=} \PY{n+nb+bp}{self}\PY{o}{.}\PY{n}{prob\PYZus{}flip}\PY{p}{(}\PY{n}{e\PYZus{}current}\PY{p}{,} \PY{n}{e\PYZus{}flip}\PY{p}{)}
                          \PY{k}{if} \PY{p}{(}\PY{n}{np}\PY{o}{.}\PY{n}{random}\PY{o}{.}\PY{n}{random}\PY{p}{(}\PY{p}{)} \PY{o}{\PYZlt{}} \PY{n}{probability\PYZus{}for\PYZus{}flip}\PY{p}{)}\PY{p}{:}
                              \PY{n}{new\PYZus{}grid}\PY{p}{[}\PY{n}{i}\PY{p}{]}\PY{p}{[}\PY{n}{j}\PY{p}{]} \PY{o}{*}\PY{o}{=} \PY{o}{\PYZhy{}}\PY{l+m+mi}{1}
                  \PY{n+nb+bp}{self}\PY{o}{.}\PY{n}{cells} \PY{o}{=} \PY{n}{new\PYZus{}grid}
          \PY{c+c1}{\PYZsh{}         raise NotImplementedError()}
                                  
                                  
              \PY{k}{def} \PY{n+nf}{magnetisation}\PY{p}{(}\PY{n+nb+bp}{self}\PY{p}{,} \PY{n}{grid}\PY{p}{)}\PY{p}{:}
                  \PY{l+s+sd}{\PYZsq{}\PYZsq{}\PYZsq{}This function calculates the mean magnetisation of all the spin in the grid}
          \PY{l+s+sd}{        returns: the mean magnetisation M\PYZsq{}\PYZsq{}\PYZsq{}}
                  \PY{c+c1}{\PYZsh{} YOUR CODE HERE}
                  \PY{n}{count} \PY{o}{=} \PY{l+m+mi}{0}
                  \PY{k}{for} \PY{n}{i} \PY{o+ow}{in} \PY{n}{grid}\PY{p}{:}
                      \PY{k}{for} \PY{n}{j} \PY{o+ow}{in} \PY{n}{i}\PY{p}{:}
                          \PY{n}{count} \PY{o}{+}\PY{o}{=} \PY{n}{j}
                  \PY{n}{M} \PY{o}{=} \PY{n}{count} \PY{o}{/} \PY{n+nb}{len}\PY{p}{(}\PY{n}{grid}\PY{p}{)} \PY{o}{/} \PY{n+nb}{len}\PY{p}{(}\PY{n}{grid}\PY{p}{[}\PY{l+m+mi}{0}\PY{p}{]}\PY{p}{)}
          \PY{c+c1}{\PYZsh{}         raise NotImplementedError()}
                  \PY{k}{return} \PY{n}{M}
              
              \PY{k}{def} \PY{n+nf}{do\PYZus{}sweeps}\PY{p}{(}\PY{n+nb+bp}{self}\PY{p}{,} \PY{n}{n\PYZus{}therm}\PY{p}{,} \PY{n}{n\PYZus{}measure}\PY{p}{)}\PY{p}{:}
                  \PY{l+s+sd}{\PYZsq{}\PYZsq{}\PYZsq{}This function carries out n\PYZus{}therm thermalisation sweeps and n\PYZus{}measure measurement sweeps.}
          \PY{l+s+sd}{        At the end of each measurement sweep the average magnetisation is computed and recorded.}
          \PY{l+s+sd}{        returns: an array of length \PYZsq{}n\PYZus{}measure\PYZsq{} containing the recorded magnetisations for each measurement sweep.}
          \PY{l+s+sd}{        It uses the sweep function, and the magnitization function\PYZsq{}\PYZsq{}\PYZsq{}}
                  \PY{c+c1}{\PYZsh{} YOUR CODE HERE}
                  \PY{k}{for} \PY{n}{i} \PY{o+ow}{in} \PY{n+nb}{range}\PY{p}{(}\PY{n}{n\PYZus{}therm}\PY{p}{)}\PY{p}{:}
                      \PY{n+nb+bp}{self}\PY{o}{.}\PY{n}{sweep}\PY{p}{(}\PY{p}{)}
                  \PY{n}{magnetisation} \PY{o}{=} \PY{p}{[}\PY{p}{]}
                  \PY{k}{for} \PY{n}{i} \PY{o+ow}{in} \PY{n+nb}{range}\PY{p}{(}\PY{n}{n\PYZus{}measure}\PY{p}{)}\PY{p}{:}
                      \PY{n+nb+bp}{self}\PY{o}{.}\PY{n}{sweep}\PY{p}{(}\PY{p}{)}
                      \PY{n}{magnetisation}\PY{o}{.}\PY{n}{append}\PY{p}{(}\PY{n+nb+bp}{self}\PY{o}{.}\PY{n}{magnetisation}\PY{p}{(}\PY{n+nb+bp}{self}\PY{o}{.}\PY{n}{cells}\PY{p}{)}\PY{p}{)}
                  \PY{k}{return} \PY{n}{magnetisation}
                                                                              
\end{Verbatim}


    You can use the cells below to test your implementation

    \hypertarget{test-and-assignment-grids}{%
\subsubsection{Test and assignment
grids}\label{test-and-assignment-grids}}

The cell below loads the test grids, \texttt{test\_grid\_1} and
\texttt{test\_grid\_2}, and their corresponding values for \(\beta\),
\texttt{test\_beta\_1} and \texttt{test\_beta\_2}.

In addition, it loads \texttt{assignement\_grid} and
\texttt{assignment\_beta}.

Use the first two grids to test your implementation. Use the third grid
to answer the assignment questions.

    \begin{Verbatim}[commandchars=\\\{\}]
{\color{incolor}In [{\color{incolor}157}]:} \PY{n}{filename} \PY{o}{=} \PY{l+s+s1}{\PYZsq{}}\PY{l+s+s1}{test\PYZus{}grid\PYZus{}1.pickle}\PY{l+s+s1}{\PYZsq{}}
          \PY{n}{f} \PY{o}{=} \PY{n+nb}{open}\PY{p}{(}\PY{n}{filename}\PY{p}{,} \PY{l+s+s1}{\PYZsq{}}\PY{l+s+s1}{rb}\PY{l+s+s1}{\PYZsq{}}\PY{p}{)}
          \PY{p}{(}\PY{n}{test\PYZus{}grid\PYZus{}1}\PY{p}{,} \PY{n}{test\PYZus{}beta\PYZus{}1}\PY{p}{)} \PY{o}{=} \PY{n}{pickle}\PY{o}{.}\PY{n}{load}\PY{p}{(}\PY{n}{f}\PY{p}{)}
          \PY{n}{f}\PY{o}{.}\PY{n}{close}\PY{p}{(}\PY{p}{)}
          
          \PY{n}{filename} \PY{o}{=} \PY{l+s+s1}{\PYZsq{}}\PY{l+s+s1}{test\PYZus{}grid\PYZus{}2.pickle}\PY{l+s+s1}{\PYZsq{}}
          \PY{n}{f} \PY{o}{=} \PY{n+nb}{open}\PY{p}{(}\PY{n}{filename}\PY{p}{,} \PY{l+s+s1}{\PYZsq{}}\PY{l+s+s1}{rb}\PY{l+s+s1}{\PYZsq{}}\PY{p}{)}
          \PY{p}{(}\PY{n}{test\PYZus{}grid\PYZus{}2}\PY{p}{,} \PY{n}{test\PYZus{}beta\PYZus{}2}\PY{p}{)} \PY{o}{=} \PY{n}{pickle}\PY{o}{.}\PY{n}{load}\PY{p}{(}\PY{n}{f}\PY{p}{)}
          \PY{n}{f}\PY{o}{.}\PY{n}{close}\PY{p}{(}\PY{p}{)}
          
          \PY{n}{filename} \PY{o}{=} \PY{l+s+s1}{\PYZsq{}}\PY{l+s+s1}{assignment\PYZus{}grid.pickle}\PY{l+s+s1}{\PYZsq{}}
          \PY{n}{f} \PY{o}{=} \PY{n+nb}{open}\PY{p}{(}\PY{n}{filename}\PY{p}{,} \PY{l+s+s1}{\PYZsq{}}\PY{l+s+s1}{rb}\PY{l+s+s1}{\PYZsq{}}\PY{p}{)}
          \PY{p}{(}\PY{n}{assignment\PYZus{}grid}\PY{p}{,} \PY{n}{assignment\PYZus{}beta}\PY{p}{)} \PY{o}{=} \PY{n}{pickle}\PY{o}{.}\PY{n}{load}\PY{p}{(}\PY{n}{f}\PY{p}{)}
          \PY{n}{f}\PY{o}{.}\PY{n}{close}\PY{p}{(}\PY{p}{)}
          
          \PY{n+nb}{print}\PY{p}{(}\PY{l+s+s2}{\PYZdq{}}\PY{l+s+s2}{ grid 1 loaded, size=}\PY{l+s+s2}{\PYZdq{}}\PY{p}{,} \PY{n+nb}{len}\PY{p}{(}\PY{n}{test\PYZus{}grid\PYZus{}1}\PY{p}{)}\PY{p}{,}\PY{l+s+s2}{\PYZdq{}}\PY{l+s+s2}{ , beta= }\PY{l+s+s2}{\PYZdq{}}\PY{p}{,} \PY{n}{test\PYZus{}beta\PYZus{}1}\PY{p}{)}
          \PY{n+nb}{print}\PY{p}{(}\PY{l+s+s2}{\PYZdq{}}\PY{l+s+s2}{ grid 2 loaded, size=}\PY{l+s+s2}{\PYZdq{}}\PY{p}{,} \PY{n+nb}{len}\PY{p}{(}\PY{n}{test\PYZus{}grid\PYZus{}2}\PY{p}{)}\PY{p}{,}\PY{l+s+s2}{\PYZdq{}}\PY{l+s+s2}{ , beta= }\PY{l+s+s2}{\PYZdq{}}\PY{p}{,} \PY{n}{test\PYZus{}beta\PYZus{}2}\PY{p}{)}
          \PY{n+nb}{print}\PY{p}{(}\PY{l+s+s2}{\PYZdq{}}\PY{l+s+s2}{ grid 3 loaded, size=}\PY{l+s+s2}{\PYZdq{}}\PY{p}{,} \PY{n+nb}{len}\PY{p}{(}\PY{n}{assignment\PYZus{}grid}\PY{p}{)}\PY{p}{,}\PY{l+s+s2}{\PYZdq{}}\PY{l+s+s2}{ , beta= }\PY{l+s+s2}{\PYZdq{}}\PY{p}{,} \PY{n}{assignment\PYZus{}beta}\PY{p}{)}
\end{Verbatim}


    \begin{Verbatim}[commandchars=\\\{\}]
 grid 1 loaded, size= 32  , beta=  1.6
 grid 2 loaded, size= 32  , beta=  0.8
 grid 3 loaded, size= 32  , beta=  0.1

    \end{Verbatim}

    \hypertarget{interaction-energy-calculation}{%
\paragraph{2. Interaction energy
calculation}\label{interaction-energy-calculation}}

The cell below allows you to test your interaction energy calculation
method. If it does not return an error your implementation might be
correct.

    \begin{Verbatim}[commandchars=\\\{\}]
{\color{incolor}In [{\color{incolor}136}]:} \PY{n}{g1} \PY{o}{=} \PY{n}{Grid}\PY{p}{(}\PY{n+nb}{len}\PY{p}{(}\PY{n}{test\PYZus{}grid\PYZus{}1}\PY{p}{)}\PY{p}{,} \PY{n}{test\PYZus{}beta\PYZus{}1}\PY{p}{)}
          \PY{n}{g2} \PY{o}{=} \PY{n}{Grid}\PY{p}{(}\PY{n+nb}{len}\PY{p}{(}\PY{n}{test\PYZus{}grid\PYZus{}2}\PY{p}{)}\PY{p}{,} \PY{n}{test\PYZus{}beta\PYZus{}2}\PY{p}{)}
          \PY{n}{cells} \PY{o}{=} \PY{p}{[}\PY{p}{(}\PY{l+m+mi}{6}\PY{p}{,}\PY{l+m+mi}{6}\PY{p}{)}\PY{p}{,} \PY{p}{(}\PY{l+m+mi}{15}\PY{p}{,}\PY{l+m+mi}{0}\PY{p}{)}\PY{p}{,} \PY{p}{(}\PY{l+m+mi}{31}\PY{p}{,}\PY{l+m+mi}{17}\PY{p}{)}\PY{p}{]}
          \PY{n}{energies\PYZus{}1} \PY{o}{=} \PY{p}{[}\PY{p}{(}\PY{l+m+mf}{0.0}\PY{p}{,} \PY{l+m+mf}{6.4}\PY{p}{)}\PY{p}{,} \PY{p}{(}\PY{l+m+mf}{0.0}\PY{p}{,} \PY{l+m+mf}{6.4}\PY{p}{)}\PY{p}{,} \PY{p}{(}\PY{l+m+mf}{0.0}\PY{p}{,} \PY{l+m+mf}{6.4}\PY{p}{)}\PY{p}{]}
          \PY{n}{energies\PYZus{}2} \PY{o}{=} \PY{p}{[}\PY{p}{(}\PY{l+m+mf}{0.8}\PY{p}{,} \PY{l+m+mf}{2.4000000000000004}\PY{p}{)}\PY{p}{,} \PY{p}{(}\PY{l+m+mf}{1.6}\PY{p}{,} \PY{l+m+mf}{1.6}\PY{p}{)}\PY{p}{,} \PY{p}{(}\PY{l+m+mf}{2.4}\PY{p}{,} \PY{l+m+mf}{0.8}\PY{p}{)}\PY{p}{]}
          
          \PY{k}{for} \PY{n}{c}\PY{p}{,} \PY{n}{cell} \PY{o+ow}{in} \PY{n+nb}{enumerate}\PY{p}{(}\PY{n}{cells}\PY{p}{)}\PY{p}{:}
              \PY{n}{i} \PY{o}{=} \PY{n}{cell}\PY{p}{[}\PY{l+m+mi}{0}\PY{p}{]}
              \PY{n}{j} \PY{o}{=} \PY{n}{cell}\PY{p}{[}\PY{l+m+mi}{1}\PY{p}{]}
              \PY{n}{e\PYZus{}1} \PY{o}{=} \PY{n}{g1}\PY{o}{.}\PY{n}{energy}\PY{p}{(}\PY{n}{i}\PY{p}{,} \PY{n}{j}\PY{p}{,} \PY{n}{test\PYZus{}beta\PYZus{}1}\PY{p}{,} \PY{n}{test\PYZus{}grid\PYZus{}1}\PY{p}{)}
              \PY{n}{e\PYZus{}2} \PY{o}{=} \PY{n}{g2}\PY{o}{.}\PY{n}{energy}\PY{p}{(}\PY{n}{i}\PY{p}{,} \PY{n}{j}\PY{p}{,} \PY{n}{test\PYZus{}beta\PYZus{}2}\PY{p}{,} \PY{n}{test\PYZus{}grid\PYZus{}2}\PY{p}{)}
              \PY{k}{assert}\PY{p}{(}\PY{n}{np}\PY{o}{.}\PY{n}{isclose}\PY{p}{(}\PY{n}{e\PYZus{}1}\PY{p}{,} \PY{n}{energies\PYZus{}1}\PY{p}{[}\PY{n}{c}\PY{p}{]}\PY{p}{)}\PY{o}{.}\PY{n}{all}\PY{p}{(}\PY{p}{)}\PY{p}{)}
              \PY{k}{assert}\PY{p}{(}\PY{n}{np}\PY{o}{.}\PY{n}{isclose}\PY{p}{(}\PY{n}{e\PYZus{}2}\PY{p}{,} \PY{n}{energies\PYZus{}2}\PY{p}{[}\PY{n}{c}\PY{p}{]}\PY{p}{)}\PY{o}{.}\PY{n}{all}\PY{p}{(}\PY{p}{)}\PY{p}{)}
          
          \PY{n+nb}{print}\PY{p}{(}\PY{l+s+s2}{\PYZdq{}}\PY{l+s+s2}{Your implementation might be correct!}\PY{l+s+s2}{\PYZdq{}}\PY{p}{)}
\end{Verbatim}


    \begin{Verbatim}[commandchars=\\\{\}]
Your implementation might be correct!

    \end{Verbatim}

    \hypertarget{probability-calculation}{%
\paragraph{3. Probability calculation}\label{probability-calculation}}

The cell below allows you to test your probability calculation method.
If it does not return an error your implementation might be correct.

    \begin{Verbatim}[commandchars=\\\{\}]
{\color{incolor}In [{\color{incolor}137}]:} \PY{n}{g1} \PY{o}{=} \PY{n}{Grid}\PY{p}{(}\PY{n+nb}{len}\PY{p}{(}\PY{n}{test\PYZus{}grid\PYZus{}1}\PY{p}{)}\PY{p}{,} \PY{n}{test\PYZus{}beta\PYZus{}1}\PY{p}{)}
          \PY{n}{energies} \PY{o}{=} \PY{p}{[}\PY{p}{(}\PY{l+m+mf}{0.1}\PY{p}{,} \PY{l+m+mf}{0.3}\PY{p}{)}\PY{p}{,} \PY{p}{(}\PY{l+m+mf}{0.2}\PY{p}{,} \PY{l+m+mf}{0.2}\PY{p}{)}\PY{p}{,} \PY{p}{(}\PY{l+m+mf}{0.3}\PY{p}{,} \PY{l+m+mf}{0.1}\PY{p}{)}\PY{p}{,} \PY{p}{(}\PY{l+m+mf}{1.5}\PY{p}{,} \PY{l+m+mf}{1.6}\PY{p}{)}\PY{p}{,} \PY{p}{(}\PY{l+m+mf}{0.1}\PY{p}{,} \PY{l+m+mf}{1.6}\PY{p}{)}\PY{p}{,} \PY{p}{(}\PY{l+m+mf}{0.8}\PY{p}{,} \PY{l+m+mf}{2.4}\PY{p}{)}\PY{p}{]}
          \PY{n}{probabilities} \PY{o}{=} \PY{p}{[}\PY{l+m+mf}{0.8187307530779818}\PY{p}{,} \PY{l+m+mi}{1}\PY{p}{,} \PY{l+m+mi}{1}\PY{p}{,} \PY{l+m+mf}{0.9048374180359595}\PY{p}{,} \PY{l+m+mf}{0.22313016014842982}\PY{p}{,} \PY{l+m+mf}{0.20189651799465544}\PY{p}{]}
          \PY{n}{this\PYZus{}prob} \PY{o}{=} \PY{p}{[}\PY{p}{]}
          
          \PY{k}{for} \PY{n}{i}\PY{p}{,} \PY{n}{e} \PY{o+ow}{in} \PY{n+nb}{enumerate}\PY{p}{(}\PY{n}{energies}\PY{p}{)}\PY{p}{:}
              \PY{n}{this\PYZus{}prob}\PY{o}{.}\PY{n}{append}\PY{p}{(}\PY{n}{g1}\PY{o}{.}\PY{n}{prob\PYZus{}flip}\PY{p}{(}\PY{n}{e}\PY{p}{[}\PY{l+m+mi}{0}\PY{p}{]}\PY{p}{,} \PY{n}{e}\PY{p}{[}\PY{l+m+mi}{1}\PY{p}{]}\PY{p}{)}\PY{p}{)}
          \PY{k}{assert}\PY{p}{(}\PY{n}{np}\PY{o}{.}\PY{n}{isclose}\PY{p}{(}\PY{n}{this\PYZus{}prob}\PY{p}{,} \PY{n}{probabilities}\PY{p}{)}\PY{o}{.}\PY{n}{all}\PY{p}{(}\PY{p}{)}\PY{p}{)}
          
          \PY{n+nb}{print}\PY{p}{(}\PY{l+s+s2}{\PYZdq{}}\PY{l+s+s2}{Your implementation might be correct!}\PY{l+s+s2}{\PYZdq{}}\PY{p}{)}
\end{Verbatim}


    \begin{Verbatim}[commandchars=\\\{\}]
Your implementation might be correct!

    \end{Verbatim}

    \hypertarget{magnetisation-calculation}{%
\paragraph{4. Magnetisation
calculation}\label{magnetisation-calculation}}

The cell below allows you to test your magnetisation method. If it does
not return an error your implementation might be correct.

    \begin{Verbatim}[commandchars=\\\{\}]
{\color{incolor}In [{\color{incolor}138}]:} \PY{n}{g1} \PY{o}{=} \PY{n}{Grid}\PY{p}{(}\PY{n+nb}{len}\PY{p}{(}\PY{n}{test\PYZus{}grid\PYZus{}1}\PY{p}{)}\PY{p}{,} \PY{n}{test\PYZus{}beta\PYZus{}1}\PY{p}{)}
          \PY{k}{assert}\PY{p}{(}\PY{n}{np}\PY{o}{.}\PY{n}{isclose}\PY{p}{(}\PY{n}{g1}\PY{o}{.}\PY{n}{magnetisation}\PY{p}{(}\PY{n}{test\PYZus{}grid\PYZus{}1}\PY{p}{)}\PY{p}{,}\PY{l+m+mf}{0.193359375}\PY{p}{)}\PY{p}{)}
          \PY{n}{g2} \PY{o}{=} \PY{n}{Grid}\PY{p}{(}\PY{n+nb}{len}\PY{p}{(}\PY{n}{test\PYZus{}grid\PYZus{}2}\PY{p}{)}\PY{p}{,} \PY{n}{test\PYZus{}beta\PYZus{}2}\PY{p}{)}
          \PY{k}{assert}\PY{p}{(}\PY{n}{np}\PY{o}{.}\PY{n}{isclose}\PY{p}{(}\PY{n}{g2}\PY{o}{.}\PY{n}{magnetisation}\PY{p}{(}\PY{n}{test\PYZus{}grid\PYZus{}2}\PY{p}{)}\PY{p}{,}\PY{o}{\PYZhy{}}\PY{l+m+mf}{0.3203125}\PY{p}{)}\PY{p}{)}
          
          \PY{n+nb}{print}\PY{p}{(}\PY{l+s+s2}{\PYZdq{}}\PY{l+s+s2}{Your implementation might be correct!}\PY{l+s+s2}{\PYZdq{}}\PY{p}{)}
\end{Verbatim}


    \begin{Verbatim}[commandchars=\\\{\}]
Your implementation might be correct!

    \end{Verbatim}

    \hypertarget{the-following-hidden-cell-uses-the-assignment-grid-to-test}{%
\paragraph{The following hidden cell uses the assignment grid to
test}\label{the-following-hidden-cell-uses-the-assignment-grid-to-test}}

\begin{itemize}
\tightlist
\item
  the calculation of energies ** 2 marks**
\item
  the calculation of the probabilities ** 2 marks **
\item
  the calculation of the magnetization ** 2 marks **
\end{itemize}

    \hypertarget{implement-the-red-black-sweep}{%
\paragraph{5 Implement the red-black
sweep}\label{implement-the-red-black-sweep}}

    \hypertarget{implement-the-thermalization-step}{%
\paragraph{6 Implement the thermalization
step}\label{implement-the-thermalization-step}}

    \hypertarget{thermalisation}{%
\paragraph{7. Thermalisation}\label{thermalisation}}

Plot the magnetisation over time for 1000 full mesh sweeps for
\(\beta = 0.1, 0.8\) and \(1.6\) (include the thermalisation period in
the plot). Include this plot in the PDF you hand in. Save the plot as
file `Thermalisation.pdf'

\textbf{4 marks}

    \begin{Verbatim}[commandchars=\\\{\}]
{\color{incolor}In [{\color{incolor}160}]:} \PY{c+c1}{\PYZsh{} for each value of beta=[0.1, 0.8, 1.6]}
          \PY{c+c1}{\PYZsh{}    generate a grid of size=32, with the given value of beta}
          \PY{c+c1}{\PYZsh{}    perform N=1000 red\PYZhy{}black sweeps (each sweep runs over the full 32x32 grid)}
          \PY{c+c1}{\PYZsh{}    calculate the mean magnetization, M for each sweep}
          \PY{c+c1}{\PYZsh{}    Plot M as a function of sweep number.}
          \PY{c+c1}{\PYZsh{}    You may want to use some of the plotting commands below.}
          \PY{n}{betas} \PY{o}{=} \PY{p}{[}\PY{l+m+mf}{0.1}\PY{p}{,} \PY{l+m+mf}{0.8}\PY{p}{,} \PY{l+m+mf}{1.6}\PY{p}{]}
          \PY{n}{size} \PY{o}{=} \PY{l+m+mi}{32}
          \PY{n}{grids} \PY{o}{=} \PY{p}{[}\PY{p}{]}
          \PY{n}{mags} \PY{o}{=} \PY{p}{[}\PY{p}{]}
          \PY{n}{tmp} \PY{o}{=} \PY{p}{[}\PY{p}{]}
          \PY{k}{for} \PY{n}{beta} \PY{o+ow}{in} \PY{n}{betas}\PY{p}{:}
              \PY{n}{grids}\PY{o}{.}\PY{n}{append}\PY{p}{(}\PY{n}{Grid}\PY{p}{(}\PY{n}{size}\PY{p}{,} \PY{n}{beta}\PY{p}{)}\PY{p}{)}
          \PY{k}{for} \PY{n}{grid} \PY{o+ow}{in} \PY{n}{grids}\PY{p}{:}
              \PY{k}{for} \PY{n}{i} \PY{o+ow}{in} \PY{n+nb}{range}\PY{p}{(}\PY{l+m+mi}{1000}\PY{p}{)}\PY{p}{:}
                  \PY{n}{grid}\PY{o}{.}\PY{n}{sweep}\PY{p}{(}\PY{p}{)}
                  \PY{n}{tmp}\PY{o}{.}\PY{n}{append}\PY{p}{(}\PY{n}{grid}\PY{o}{.}\PY{n}{magnetisation}\PY{p}{(}\PY{n}{grid}\PY{o}{.}\PY{n}{cells}\PY{p}{)}\PY{p}{)}
              \PY{n}{mags}\PY{o}{.}\PY{n}{append}\PY{p}{(}\PY{n}{tmp}\PY{p}{)}
              \PY{n}{tmp} \PY{o}{=} \PY{p}{[}\PY{p}{]}
          \PY{c+c1}{\PYZsh{} set\PYZhy{}up the figure}
          \PY{n+nb}{print}\PY{p}{(}\PY{l+s+s2}{\PYZdq{}}\PY{l+s+s2}{Calculation finished}\PY{l+s+s2}{\PYZdq{}}\PY{p}{)}
          \PY{n}{fig}\PY{p}{,} \PY{n}{ax} \PY{o}{=} \PY{n}{plt}\PY{o}{.}\PY{n}{subplots}\PY{p}{(}\PY{l+m+mi}{1}\PY{p}{,}\PY{l+m+mi}{1}\PY{p}{,} \PY{n}{figsize} \PY{o}{=} \PY{p}{(}\PY{l+m+mi}{8}\PY{p}{,} \PY{l+m+mi}{5}\PY{p}{)}\PY{p}{)}
          \PY{n}{file} \PY{o}{=} \PY{l+s+s2}{\PYZdq{}}\PY{l+s+s2}{Thermalisation.pdf}\PY{l+s+s2}{\PYZdq{}}
          \PY{c+c1}{\PYZsh{} caculate mag, the average magnetization, for N=1000 sweeps}
          \PY{c+c1}{\PYZsh{} \PYZsh{} YOUR CODE HERE}
          \PY{c+c1}{\PYZsh{} raise NotImplementedError()}
          \PY{c+c1}{\PYZsh{} plot the result, annotate the file, and save the file}
          \PY{n}{ax}\PY{o}{.}\PY{n}{set\PYZus{}xlabel}\PY{p}{(}\PY{l+s+sa}{r}\PY{l+s+s1}{\PYZsq{}}\PY{l+s+s1}{\PYZdl{}N\PYZus{}}\PY{l+s+si}{\PYZob{}steps\PYZcb{}}\PY{l+s+s1}{\PYZdl{}}\PY{l+s+s1}{\PYZsq{}}\PY{p}{)}
          \PY{n}{ax}\PY{o}{.}\PY{n}{set\PYZus{}ylabel}\PY{p}{(}\PY{l+s+sa}{r}\PY{l+s+s1}{\PYZsq{}}\PY{l+s+s1}{\PYZdl{}M\PYZdl{}}\PY{l+s+s1}{\PYZsq{}}\PY{p}{)}
          \PY{n}{ax}\PY{o}{.}\PY{n}{set\PYZus{}ylim}\PY{p}{(}\PY{p}{[}\PY{o}{\PYZhy{}}\PY{l+m+mf}{1.05}\PY{p}{,} \PY{l+m+mf}{1.05}\PY{p}{]}\PY{p}{)}
          \PY{n}{i} \PY{o}{=} \PY{l+m+mi}{0}
          \PY{k}{for} \PY{n}{mag} \PY{o+ow}{in} \PY{n}{mags}\PY{p}{:}
              \PY{c+c1}{\PYZsh{} pass the value of beta into the plot command to generate the label, as in}
              \PY{n}{ax}\PY{o}{.}\PY{n}{plot}\PY{p}{(}\PY{n}{np}\PY{o}{.}\PY{n}{arange}\PY{p}{(}\PY{n+nb}{len}\PY{p}{(}\PY{n}{mag}\PY{p}{)}\PY{p}{)}\PY{p}{,} \PY{n}{mag}\PY{p}{,} \PY{n}{label}\PY{o}{=}\PY{l+s+s1}{\PYZsq{}}\PY{l+s+s1}{beta=}\PY{l+s+si}{\PYZpc{}.2f}\PY{l+s+s1}{\PYZsq{}}\PY{o}{\PYZpc{}}\PY{k}{betas}[i])
              \PY{n}{i} \PY{o}{+}\PY{o}{=} \PY{l+m+mi}{1}
          \PY{n}{ax}\PY{o}{.}\PY{n}{legend}\PY{p}{(}\PY{p}{)}
          \PY{n}{plt}\PY{o}{.}\PY{n}{savefig}\PY{p}{(}\PY{n}{file}\PY{p}{)}
          \PY{n}{fig}\PY{o}{.}\PY{n}{show}\PY{p}{(}\PY{p}{)}
\end{Verbatim}


    \begin{Verbatim}[commandchars=\\\{\}]
Calculation finished

    \end{Verbatim}

    \begin{Verbatim}[commandchars=\\\{\}]
/opt/conda/lib/python3.6/site-packages/matplotlib/figure.py:457: UserWarning: matplotlib is currently using a non-GUI backend, so cannot show the figure
  "matplotlib is currently using a non-GUI backend, "

    \end{Verbatim}

    \begin{center}
    \adjustimage{max size={0.9\linewidth}{0.9\paperheight}}{output_33_2.png}
    \end{center}
    { \hspace*{\fill} \\}
    
    \hypertarget{ensemble-averaged-magnetisation-bar-m-as-a-function-of-beta}{%
\paragraph{\texorpdfstring{8. Ensemble-averaged magnetisation,
\(\bar M\), as a function of
\(\beta\)}{8. Ensemble-averaged magnetisation, \textbackslash{}bar M, as a function of \textbackslash{}beta}}\label{ensemble-averaged-magnetisation-bar-m-as-a-function-of-beta}}

Once the system has thermalized (a good choice is
\(N_\mathrm{therm} =400\) thermalisation sweeps), measure the
time-averaged magnetisation over 1000 sweeps. From this, estimate the
ensemble-averaged magnetisation, \(\bar M\). Plot \(\bar M\) as a
function of \(1/\beta\) for
\(\beta = 1.6, 1.3, 1.1, 1.0, 0.9, 0.8, 0.7, 0.6, 0.4, 0.1\). Include
this plot in the PDF you hand in. Save the plot to a file
`Magnetisation.pdf'.

Perform the following steps: - for each of the listed values of beta: -
create a \(32\times32\) random grid - sweep the for N=400 initial
`thermalization' steps - \textbf{then} sweep for another 1000 steps,
calculating \(M\) after every sweep - use this to compute the
ensemble-averaged magnitization, \(\bar M\) for that value of \(\beta\)
- plot \(\bar M\) as a function of \(\beta^{-1}\)

\textbf{4 marks}

    \begin{Verbatim}[commandchars=\\\{\}]
{\color{incolor}In [{\color{incolor}162}]:} \PY{c+c1}{\PYZsh{} Step 8: Magnetisation}
          
          
          \PY{c+c1}{\PYZsh{} set\PYZhy{}up the figure}
          \PY{n}{file} \PY{o}{=}  \PY{l+s+s2}{\PYZdq{}}\PY{l+s+s2}{Magnetisation.pdf}\PY{l+s+s2}{\PYZdq{}}  
          \PY{n}{fig}\PY{p}{,} \PY{n}{ax} \PY{o}{=} \PY{n}{plt}\PY{o}{.}\PY{n}{subplots}\PY{p}{(}\PY{l+m+mi}{1}\PY{p}{,}\PY{l+m+mi}{1}\PY{p}{,} \PY{n}{figsize} \PY{o}{=} \PY{p}{(}\PY{l+m+mi}{8}\PY{p}{,} \PY{l+m+mi}{5}\PY{p}{)}\PY{p}{)}
          
          \PY{c+c1}{\PYZsh{} the range of values of beta}
          \PY{n}{size} \PY{o}{=} \PY{l+m+mi}{32}
          \PY{n}{betas} \PY{o}{=} \PY{p}{[}\PY{l+m+mf}{1.6}\PY{p}{,}\PY{l+m+mf}{1.3}\PY{p}{,}\PY{l+m+mf}{1.1}\PY{p}{,}\PY{l+m+mf}{1.0}\PY{p}{,}\PY{l+m+mf}{0.9}\PY{p}{,}\PY{l+m+mf}{0.8}\PY{p}{,}\PY{l+m+mf}{0.7}\PY{p}{,}\PY{l+m+mf}{0.6}\PY{p}{,}\PY{l+m+mf}{0.4}\PY{p}{,}\PY{l+m+mf}{0.1}\PY{p}{]}
          \PY{n}{mean\PYZus{}mags} \PY{o}{=} \PY{p}{[}\PY{p}{]}
          \PY{c+c1}{\PYZsh{} Loop over values of beta, computing the ensemble\PYZhy{}averaged M for each beta}
          \PY{c+c1}{\PYZsh{}    name the resulting ensemble\PYZhy{}averaged M mean\PYZus{}mags}
          \PY{c+c1}{\PYZsh{}    It is an array with the same dimension as betas}
          \PY{k}{for} \PY{n}{beta} \PY{o+ow}{in} \PY{n}{betas}\PY{p}{:}
              \PY{n}{grid} \PY{o}{=} \PY{n}{Grid}\PY{p}{(}\PY{n}{size}\PY{p}{,} \PY{n}{beta}\PY{p}{)}
              \PY{n}{mean\PYZus{}mags}\PY{o}{.}\PY{n}{append}\PY{p}{(}\PY{n}{np}\PY{o}{.}\PY{n}{mean}\PY{p}{(}\PY{n}{grid}\PY{o}{.}\PY{n}{do\PYZus{}sweeps}\PY{p}{(}\PY{l+m+mi}{400}\PY{p}{,} \PY{l+m+mi}{1000}\PY{p}{)}\PY{p}{)}\PY{p}{)}
          \PY{c+c1}{\PYZsh{} YOUR CODE HERE}
          
          \PY{c+c1}{\PYZsh{} make the plot}
          \PY{n}{ax}\PY{o}{.}\PY{n}{set\PYZus{}xlabel}\PY{p}{(}\PY{l+s+sa}{r}\PY{l+s+s1}{\PYZsq{}}\PY{l+s+s1}{\PYZdl{}}\PY{l+s+s1}{\PYZbs{}}\PY{l+s+s1}{beta\PYZca{}}\PY{l+s+s1}{\PYZob{}}\PY{l+s+s1}{\PYZhy{}1\PYZcb{}\PYZdl{}}\PY{l+s+s1}{\PYZsq{}}\PY{p}{)}
          \PY{n}{ax}\PY{o}{.}\PY{n}{set\PYZus{}ylabel}\PY{p}{(}\PY{l+s+sa}{r}\PY{l+s+s1}{\PYZsq{}}\PY{l+s+s1}{\PYZdl{}}\PY{l+s+s1}{\PYZbs{}}\PY{l+s+s1}{bar M\PYZdl{}}\PY{l+s+s1}{\PYZsq{}}\PY{p}{)}
          \PY{n}{ax}\PY{o}{.}\PY{n}{set\PYZus{}ylim}\PY{p}{(}\PY{p}{[}\PY{o}{\PYZhy{}}\PY{l+m+mf}{1.05}\PY{p}{,} \PY{l+m+mf}{1.05}\PY{p}{]}\PY{p}{)}
          \PY{n}{ax}\PY{o}{.}\PY{n}{set\PYZus{}xlim}\PY{p}{(}\PY{l+m+mi}{0}\PY{p}{,}\PY{l+m+mi}{4}\PY{p}{)}
          \PY{n}{plt}\PY{o}{.}\PY{n}{plot}\PY{p}{(}\PY{l+m+mf}{1.}\PY{o}{/}\PY{n}{np}\PY{o}{.}\PY{n}{array}\PY{p}{(}\PY{n}{betas}\PY{p}{)}\PY{p}{,} \PY{n}{mean\PYZus{}mags}\PY{p}{)}
          \PY{n}{plt}\PY{o}{.}\PY{n}{savefig}\PY{p}{(}\PY{n}{file}\PY{p}{)}
          \PY{n}{fig}\PY{o}{.}\PY{n}{show}\PY{p}{(}\PY{p}{)}
\end{Verbatim}


    \begin{Verbatim}[commandchars=\\\{\}]
/opt/conda/lib/python3.6/site-packages/matplotlib/figure.py:457: UserWarning: matplotlib is currently using a non-GUI backend, so cannot show the figure
  "matplotlib is currently using a non-GUI backend, "

    \end{Verbatim}

    \begin{center}
    \adjustimage{max size={0.9\linewidth}{0.9\paperheight}}{output_35_1.png}
    \end{center}
    { \hspace*{\fill} \\}
    
    \hypertarget{critical-temperature}{%
\paragraph{9. Critical temperature}\label{critical-temperature}}

The critical temperature, \(T_c\), in the mean field approximation, is
given by \[
T_c=\frac{h J}{2k_B},
\] where h is the number of nearest-neighbours, as discussed in the
lecture. Use this to calculate the critical value \(\beta_c\) which
corresponds to the critical temperature and mark it in the plot produced
in step \textbf{8} (If you could not produce the plot in step \textbf{8}
you may also mention the numerical value of \(\beta_c\) on the solution
you hand in).

Enter your answer in the box below. If you don't see a box, execute the
hidden cell below.

\textbf{2 marks}

    \begin{Verbatim}[commandchars=\\\{\}]
{\color{incolor}In [{\color{incolor}158}]:} \PY{n}{beta\PYZus{}crit}\PY{o}{=}\PY{n}{mywidgets}\PY{o}{.}\PY{n}{myFloatBox}\PY{p}{(}\PY{l+s+s1}{\PYZsq{}}\PY{l+s+s1}{Phasetransition}\PY{l+s+s1}{\PYZsq{}}\PY{p}{,}\PY{l+s+s1}{\PYZsq{}}\PY{l+s+s1}{P1}\PY{l+s+s1}{\PYZsq{}}\PY{p}{,}\PY{l+s+s1}{\PYZsq{}}\PY{l+s+s1}{beta\PYZus{}c=}\PY{l+s+s1}{\PYZsq{}}\PY{p}{,} \PY{l+s+s1}{\PYZsq{}}\PY{l+s+s1}{Enter your analytically calculated value of beta\PYZus{}c to 3 sig figs}\PY{l+s+s1}{\PYZsq{}}\PY{p}{)}
          \PY{n}{beta\PYZus{}crit}\PY{o}{.}\PY{n}{getWidget}\PY{p}{(}\PY{p}{)}
\end{Verbatim}


    
    \begin{verbatim}
VBox(children=(HTMLMath(value='Enter your analytically calculated value of beta_c to 3 sig figs', placeholder=…
    \end{verbatim}

    
    \hypertarget{mean-field-approximation}{%
\paragraph{10. Mean field
approximation}\label{mean-field-approximation}}

The ensemble-averaged value of the mean magnitization in the mean field
approximation, is the solution of \[
\bar M - \tanh\left( \frac{T_c\bar M}{T} \right) = 0\,,
\] as discussed in the lecture. This equation can not be solved
analytically for \(\bar M\), for given \(T_c/T\).

Rewrite the equation in terms of \(\beta\) using the relation between
\(T\) and \(\beta\) derived before, and solve the resulting formula
numerically for
\(\beta = 1.6, 1.3, 1.1, 1.0, 0.9, 0.8, 0.7, 0.6, 0.4, 0.1\). You may
want to use the root finding methods implemented in previous exercises.
Redo the plot of step \textbf{8}, but on top of the numerical result,
over plot the mean field approximation. Add a legend to the plot which
allows to distinguish the solution obtained in step \textbf{8} from the
mean field approximation.

The numerical value of \(\bar M\) versus \(\beta\) shows that the
transition from \(\bar M=0\) at hight \(T\) to \(\bar M>0\) at low \(T\)
is not infinitely sharp. To quantify where the transition occurs, it
might be useful to compute \(\beta_{1/2}\), the value of \(\beta\) where
\(\bar M=1/2\). Calculate this for both the numerical and mean field
approximation, and indicate the point
\((\beta,\bar M)=(\beta_{1/2}, 1/2)\) on the plot.

Save the plot as `MeanField.pdf'

\textbf{4 marks}

    \begin{Verbatim}[commandchars=\\\{\}]
{\color{incolor}In [{\color{incolor} }]:} \PY{c+c1}{\PYZsh{} Implement the mean field calculation here: calculate mean\PYZus{}mag\PYZus{}MF, the mean field approximation to the magnetisation}
        \PY{c+c1}{\PYZsh{} for a given value of beta. Also implement the calculation of the critical value, beta\PYZus{}c, according to the}
        \PY{c+c1}{\PYZsh{} MFA}
        
        \PY{c+c1}{\PYZsh{} YOUR CODE HERE}
        \PY{k}{raise} \PY{n+ne}{NotImplementedError}\PY{p}{(}\PY{p}{)}
\end{Verbatim}


    \begin{Verbatim}[commandchars=\\\{\}]
{\color{incolor}In [{\color{incolor} }]:} \PY{c+c1}{\PYZsh{} YOUR CODE HERE}
        \PY{k}{raise} \PY{n+ne}{NotImplementedError}\PY{p}{(}\PY{p}{)}
\end{Verbatim}



    % Add a bibliography block to the postdoc
    
    
    
    \end{document}
